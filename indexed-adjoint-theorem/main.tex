\documentclass[10pt, oneside]{article} 
\usepackage{mathtools, amsmath, amsthm, amssymb, calrsfs, wasysym, verbatim, bbm, color, graphics, geometry, array, enumerate}

\usepackage{eucal}[mathcal]

\usepackage[backend=biber, style=alphabetic]{biblatex}
\addbibresource{references.bib}
\usepackage{quiver}
\usepackage{bbm}
\usepackage{abstract}
\renewcommand{\abstractname}{}    % clear the title
\renewcommand{\absnamepos}{empty}

\geometry{tmargin=.75in, bmargin=.75in, lmargin=.75in, rmargin = .75in}  

\newcommand{\R}{\mathbb{R}}
\newcommand{\C}{\mathbb{C}}
\newcommand{\Z}{\mathbb{Z}}
\newcommand{\N}{\mathbb{N}}
\newcommand{\Q}{\mathbb{Q}}
\newcommand{\Cdot}{\boldsymbol{\cdot}}

\newtheorem{theorem}{Theorem}[section]
\newtheorem{definition}[theorem]{Definition}
\newtheorem{remark}[theorem]{Remark}
\newtheorem{lemma}[theorem]{Lemma}
\newtheorem{corollary}[theorem]{Corollary}
\newtheorem{example}[theorem]{Example}

\title{The indexed adjoint functor theorem}
\author{Gabriel C. Barbosa}
\date{2023}

\nocite{*}

\begin{document}

\maketitle

\tableofcontents

\vspace{.25in}

\section{Background}

We only assume familiarity with the most fundamental definitions about fibrations and indexed categories: what they are, and how they are equivalent. A reading of the first few chapters of \cite{Streicher2018} suffices.

\vspace{0.5cm}

\def\arraystretch{1.5}
\begin{tabular}{ | m{5em} | m{5cm}| m{5cm} | } 
  \hline
  & Fibrations over $B$ & $B$-indexed categories \\ 
  \hline
  $0$-cells & Grothendieck fibration $X \to B$ & Pseudofunctor $B^{op} \to \mathrm{Cat}$ \\ 
  \hline
  $1$-cells & Cartesian functor & Pseudonatural transformation \\
  \hline
  $2$-cells & Vertical transformation & Modification \\
  \hline
\end{tabular}
\vspace{0.5cm}

The direction $\implies$ takes a cleavage inducing functors $x^\ast$ which assemble naturally into a pseudofunctor. The opposite direction takes the total category / Grothendieck construction of the pseudofunctor.

For notation, we always use blackboard bold $\mathbb{C}$ for indexed categories induced by a fibration with domain $\mathcal{C}$.


\section{Locally small and well-powered categories}

To give some motivation, let $i: \mathbf{Set} \to \mathbf{Cat}$ be the discrete category functor, so for any category $\mathcal{C}$ we have an indexing induced by $X \to \mathcal{C}^{iX}$. We actually write $\mathcal{C}^I$.

Let's assume $\mathcal{C}$ is locally small. Then given families $X, Y \in \mathcal{C}^I$, there's an obvious candidate hom-set $[X, Y]$ indexed by $I$ with $[X, Y]_i = [X_i, Y_i]$. Letting $p: [X, Y] \to I$, we take $p^\ast X$. We then have a canonical indexed map $\phi: p^\ast X \to p^\ast Y$ whose $f$-component is $f$. Then $[X, Y]$ being a hom-set can be expressed by the following universal property: for any $h: J \to I$, $\psi: h^\ast X \to p^\ast Y$ we have a unique factoring depicted below

\[\begin{tikzcd}[ampersand replacement=\&]
	{h^\ast X} \&\& {p^\ast X} \&\& J \&\& {[X, Y]} \\
	\& Y \&\&\&\& I
	\arrow[dashed, from=1-1, to=1-3]
	\arrow["\psi"', from=1-1, to=2-2]
	\arrow["\phi", from=1-3, to=2-2]
	\arrow["h"', from=1-5, to=2-6]
	\arrow["p", from=1-7, to=2-6]
	\arrow[dashed, from=1-5, to=1-7]
\end{tikzcd}\]

It's pretty straightforward to see this: $\psi$ consists of $h^\ast X \to h^\ast Y$, that is, a $J$-indexed collection of maps $[X, Y]$, but this already gives us the map on the right, and consequently the map on the left.

With the above motivation in mind, consider:

\begin{definition}
Let $P: X \to B$ be a fibration.

\begin{enumerate}[i)]
    \item Given $X, Y \in X^I$, $\operatorname{Hom}_I{(X, Y)}$ is the category whose objects are pairs
    \[\begin{tikzcd}[ampersand replacement=\&]
    	\& U \\
    	X \&\& Y
    	\arrow["\varphi"', from=1-2, to=2-1]
    	\arrow["f", from=1-2, to=2-3]
    \end{tikzcd}\]
    \noindent s.t. $P(\varphi) = P(f)$ where $\varphi$ is cartesian and whose arrows are s.t.

    \[\begin{tikzcd}[ampersand replacement=\&]
	\& X \\
	U \&\& V \\
	\& Y
	\arrow["\varphi", from=2-1, to=1-2]
	\arrow["f"', from=2-1, to=3-2]
	\arrow["{\varphi^\prime}"', from=2-3, to=1-2]
	\arrow["g", from=2-3, to=3-2]
	\arrow["\theta", from=2-1, to=2-3]
    \end{tikzcd}\]

    We call $P$ locally small if the fibration $\operatorname{Hom}_I{(X, Y)} \to B/I$ is representable, that is, $\operatorname{Hom}_I{(X, Y)}$ admits a terminal object.
\end{enumerate}
\end{definition}

Have the reader read \cite{Johnstone2002}, they'll probably be familiar with a different definition:

\begin{definition}
    \begin{enumerate}[i)]
    \item[]
        \item If $I$ is any category, $\mathrm{Rect}(I, \mathbb{C})$ is the category of vertical diagrams (in the non-indexed sense, on the fibration domain) in $\mathbb{C}$ and transformations with cartesian components.

        \item We say $\mathbb{C}$ is locally small if the forgetful functor $\mathrm{Rect}(\mathbbm{2}, \mathbb{C}) \to \mathrm{Rect}(\mathbf{2}, \mathbb{C})$ has a right adjoint.

        \item More generally, we say $\mathbb{C}$ \textbf{satisfies the comprehension scheme} for $F: I \to J$ if $F^\ast: \mathrm{Rect}(J, \mathbb{C}) \to \mathrm{Rect}(I, \mathbb{C})$ has a right adjoint.
    \end{enumerate}
\end{definition}

Expanding the second definition in terms of comma categories, we have a straightforward equivalence.

\begin{definition}
    \begin{enumerate}[i)]
    \item[]
        \item We write $\mathrm{Rect}_\text{mono}(I, \mathbb{C})$ for the full subcategory of $\mathrm{Rect}(I, \mathbb{C})$ whose diagrams have all arrows mono.

        \item We say $\mathbb{C}$ is well-powered if each $x^\ast$ preserves monos and the codomain functor $\mathrm{Rect}_\text{mono}(\mathbbm{2}, \mathbb{C}) \to \mathrm{Rect}(\mathbf{1}, \mathbb{C})$ has a right adjoint.
    \end{enumerate}
\end{definition}

\section{Unpacking the properties}

We now consider what does it mean for an indexed category to be locally-small, well-powered, etc. Recall that in the Grothendieck fibration obtained from $\mathbb{C}$, cartesian arrows are precisely $(f, u)$ with $u$ an isomorphism.

\begin{lemma}
    An indexed category $\mathbb{C}$ is locally small iff for $X, Y$, we have $d: \mathrm{hom}(X, Y) \to I \times J, g: d_0^\ast X \to d_1^\ast Y$ s.t. for every $u: K \to I \times J$ and $f: u_0^\ast X \to u_1^\ast Y$ we have unique $v: K \to \mathrm{hom}(X, Y)$ such that $u = d \circ v$ and
    
    $$ f = \psi_{d, v} \circ v^\ast g \circ \psi_{d, v}^{-1} $$
\end{lemma}
\begin{proof}
    \[\begin{tikzcd}[ampersand replacement=\&]
	V \&\& X \&\&\& {u^\ast X} \\
	\& {g^\ast X} \&\&\& X \&\& Y \\
	\&\&\&\&\& {d^\ast X}
	\arrow["{(g, \alpha)}", from=1-1, to=1-3]
	\arrow["{(1, \psi_{g, 1}^{-1} \alpha)}"', from=1-1, to=2-2]
	\arrow["{(g, 1)}"', from=2-2, to=1-3]
	\arrow["{(d, 1)}", from=3-6, to=2-5]
	\arrow["{(u,1)}"', from=1-6, to=2-5]
	\arrow["{(v, k)}"{description}, dashed, from=1-6, to=3-6]
	\arrow["{(u, f)}", from=1-6, to=2-7]
	\arrow["{(d, g)}"', from=3-6, to=2-7]
\end{tikzcd}\]
\end{proof}

In the same way,

\begin{theorem}
    An indexed category whose transition functors preserve monos is well-powered if given $A \in C^I$ we have $d: \mathrm{sub}(A) \to I$ and $m: X \rightarrowtail d^\ast A$ generic in the sense that for any other $m^\prime: B \to u^\ast A$ we have $d \circ v = u$ and $v^\ast m = m^\prime$ as subobjects.
\end{theorem}

Let us consider some other properties.

\begin{definition}
    \begin{enumerate}[i)]

    \item[]
    
        \item We say $\mathbb{C}$ has definable equality if it satisfies the comprehension scheme w.r.t. the functor collapsing two parallel arrows.

        \item We say $\mathbb{C}$ has definable invertibility if it satisfies the comprehension scheme w.r.t. the functor including $\mathbbm{2}$ inside its grupoid closure.
    \end{enumerate}
\end{definition}

The following are easily proved by unpacking the right adjoint definition with comma categories:

\begin{lemma}
    \begin{enumerate}[i)]
        \item[]

        \item $\mathbb{C}$ has definable equality iff for every $f, g \in \mathcal{C}^I$ we have a subobject $I^\prime \rightarrowtail I$ s.t. every $x: J \to I$ s.t. $x^\ast f = x^\ast g$ factors through $I^\prime$.
        
        \item $\mathbb{C}$ has definable invertibility iff for every $f \in \mathcal{C}^I$ we have a subobject $I^\prime \rightarrowtail I$ s.t. every $x: J \to I$ s.t. $x^\ast f$ is iso factors through $I^\prime$.
    \end{enumerate}
\end{lemma}

\begin{theorem}\label{thm:locally-small-def-eq-inv}
    Let $\mathbb{C}$ be a $\mathcal{S}$-indexed locally small category:
    
    \begin{enumerate}[i)]
        \item If $\mathcal{S}$ has equalizers, $\mathbb{C}$ has definable equality

        \item If $\mathcal{S}$ also has pullbacks, $\mathbb{C}$ has definable invertibility
    \end{enumerate}
\end{theorem}
\begin{proof}
    i): Let $d: J \to I \times I$ be the arrow indexing morphisms given by $\mathbb{C}$ being locally small. Then both $f$ and $g$ correspond to arrows $I \to J$, and taking their equalizer we obtain the desired subobject.

    ii): Let $f: X \to Y \in \mathcal{C}^I$ be given. Let

    \[\begin{tikzcd}[ampersand replacement=\&]
	J \&\& {\mathrm{hom}(Y, X)} \\
	\\
	I \&\& {I \times I}
	\arrow[tail, from=3-1, to=3-3]
	\arrow[from=1-3, to=3-3]
	\arrow["x"', from=1-1, to=3-1]
	\arrow["v", tail, from=1-1, to=1-3]
	\arrow["\lrcorner"{anchor=center, pos=0.125}, draw=none, from=1-1, to=3-3]
\end{tikzcd}\]

    Letting $h$ be the generic such morphism, we put $g = x^\ast h$, where $h$ is the generic morphism. Then we take $K \rightarrowtail J$ to be the intersection of arrows measuring equality of $x^\ast (f) \circ g = 1, g \circ x^\ast (f)$, respectively. Then one can see that factorizations of some $y$ through $K \rightarrowtail J \rightarrow I$ correspond to inverses of $y^\ast (f)$, but by uniqueness the latter arrow is mono.
 \end{proof}

We will later use these to transfer properties from an indexed category to its comma construction.

\section{Indexed (co)completeness}

Through this entire section, we fix a base category $\mathcal{S}$ which we'll assume for our ends to be a topos (though it needn't be in a lot of results we'll state).

\begin{definition}
    We say $\mathbb{C}$ has $\mathcal{S}$-indexed products if every $x^\ast$ has a right adjoint $\Pi_X$, satisfying the following Beck-Chevalley condition: for every pullback square

    \[\begin{tikzcd}[ampersand replacement=\&]
	I \&\& K \\
	\\
	J \&\& L
	\arrow["w", from=1-1, to=1-3]
	\arrow["x"', from=1-1, to=3-1]
	\arrow["z"', from=3-1, to=3-3]
	\arrow["y", from=1-3, to=3-3]
	\arrow["\lrcorner"{anchor=center, pos=0.125}, draw=none, from=1-1, to=3-3]
\end{tikzcd}\]

    \noindent we have a $2$-cell isomorphism

    \[\begin{tikzcd}[ampersand replacement=\&]
	{C^J} \&\& {C^I} \\
	\\
	{C^L} \&\& {C^K}
	\arrow["{\Pi_y}"', from=1-1, to=3-1]
	\arrow["{\Pi_x}", from=1-3, to=3-3]
	\arrow["{w^\ast}", from=1-1, to=1-3]
	\arrow["{z^\ast}"', from=3-1, to=3-3]
	\arrow["\Phi", shorten <=22pt, shorten >=22pt, Rightarrow, from=3-1, to=1-3]
	\arrow["\cong"', draw=none, from=3-1, to=1-3]
\end{tikzcd}\]
    
    \noindent where $\Phi_A$ is the $x^\ast$-transpose of

    $$x^\ast z^\ast \Pi_y A \xrightarrow[]{\cong} w^\ast y^\ast \Pi_y A \xrightarrow[]{w^\ast \epsilon^y_A} w^\ast A$$

    We also have the dual definition of indexed coproducts.
\end{definition}

\begin{remark}\label{rem:unit-identities}
    Note that due do uniqueness of adjoints we can get coherence isos $\Theta, \psi$ for $\Pi_x$, for instance, given by

    $$\Theta^{-1} = \epsilon^1 \circ \theta_{\Pi_1}$$

    $$\psi_{y, x} = \Pi_{yx} \epsilon^x \circ \Pi_{yx} x^\ast \epsilon^y_{\Pi_x} \circ \Pi_{yx} {\phi^{-1}_{y, x}}_{\Pi_y \Pi_x} \circ \eta^{yx}_{\Pi_y \Pi_x}$$

    Which gives us

    $$(\Theta^{-1} \ast \theta^{-1}) \circ \eta^1 = \theta^{-1} \circ \epsilon_{1^\ast} \circ \theta_{\Pi_1 1^\ast} \circ \eta^1 = 1$$
    Also,
    $$\psi_{y, x} = \Pi_{yx} \epsilon^x \circ \Pi_{yx} x^\ast \epsilon^y_{\Pi_x} \circ \Pi_{yx} {\phi^{-1}_{y, x}}_{\Pi_y \Pi_x} \circ \eta^{yx}_{\Pi_y \Pi_x}$$

    so a straightforward but long and tedious calculation shows:
    
    \begin{align*}
(\psi_{y, x} \ast \phi_{y, x}) \circ \Pi_y \eta^x_{y^\ast} \circ \eta^y \\ = \Pi_{yx} \phi_{y, x} \circ [(\Pi_{yx} \epsilon^x \circ \Pi_{yx} x^\ast \epsilon^y_{\Pi_x} \circ \Pi_{yx} {\phi^{-1}_{y, x}}_{\Pi_y \Pi_x} \circ \eta^{yx}_{\Pi_y \Pi_x}) \ast (x^\ast y^\ast)] \circ \Pi_y \eta^x_{y^\ast} \circ \eta^y \\
= \Pi_{yx} \phi_{y, x} \circ [(\Pi_{yx} \epsilon^x \circ \Pi_{yx} x^\ast \epsilon^y_{\Pi_x} \circ \Pi_{yx} {\phi^{-1}_{y, x}}_{\Pi_y \Pi_x}) \ast (x^\ast y^\ast)] \circ \Pi_{yx} {(yx)}^\ast \Pi_y \eta^x_{y^\ast} \circ \eta^{yx}_{\Pi_y y^\ast} \circ \eta^y && \text{(Naturality of $\eta^{yx}$)}
\\
= \Pi_{yx} \phi_{y, x} \circ [(\Pi_{yx} \epsilon^x \circ \Pi_{yx} x^\ast \epsilon^y_{\Pi_x}) \ast (x^\ast y^\ast)] \circ \Pi_{yx} x^\ast y^\ast \Pi_y \eta^x_{y^\ast} \circ \Pi_{yx} {\phi^{-1}_{y, x}}_{\Pi_y y^\ast} \circ \eta^{yx}_{\Pi_y y^\ast} \circ \eta^y && \text{(Naturality of $\phi_{y, x}^{-1}$)}
\\
= \Pi_{yx} \phi_{y, x} \circ [(\Pi_{yx} \epsilon^x) \ast (x^\ast y^\ast)] \circ  \Pi_{yx} x^\ast \eta^x_{y^\ast} \circ \Pi_{yx} x^\ast  \epsilon^y_{y^\ast} \circ \Pi_{yx} {\phi^{-1}_{y, x}}_{\Pi_y y^\ast} \circ \eta^{yx}_{\Pi_y y^\ast} \circ \eta^y && \text{(Naturality of $\epsilon^y$)}
\\
= \Pi_{yx} \phi_{y, x} \circ \Pi_{yx} x^\ast  \epsilon^y_{y^\ast} \circ \Pi_{yx} {\phi^{-1}_{y, x}}_{\Pi_y y^\ast} \circ \eta^{yx}_{\Pi_y y^\ast} \circ \eta^y && \text{(Triangle identity for $x$)}
\\
= \Pi_{yx} \phi_{y, x} \circ \Pi_{yx} x^\ast  \epsilon^y_{y^\ast} \circ \Pi_{yx} {\phi^{-1}_{y, x}}_{\Pi_y y^\ast} \circ \Pi_{yx} {(yx)}^\ast \eta^y \circ \eta^{yx} && \text{(Naturality of $\eta^{yx}$)}
\\
= \Pi_{yx} \phi_{y, x} \circ \Pi_{yx} x^\ast  \epsilon^y_{y^\ast} \circ \Pi_{yx} x^\ast y^\ast \eta^y \circ \Pi_{yx} {\phi^{-1}_{y, x}}\circ  \eta^{yx} && \text{(Naturality of $\phi_{y, x}^{-1}$)}
\\
= \eta^{yx} && \text{(Triangle identity for $y$)}
\end{align*}
\end{remark}

A nice characterization of the existence of internal coproducts comes from the following result:

\begin{theorem}
    Let $P: \mathcal{C} \to \mathcal{S}$ be a Grothendieck fibration.

    \begin{enumerate}[i)]
	\item Each $x^\ast$ has a left adjoint iff $P$ is an opfibration as well.

        \item $\mathbb{C}$ has $\mathcal{S}$-indexed coproducts iff $P$ is an opfibration and the pullback of any cocartesian arrow along a cartesian arrow exists and is again cocartesian.
    \end{enumerate}
\end{theorem}
\begin{proof}
    \cite[Lemma B.1.4.5]{Johnstone2002}
\end{proof}

\begin{definition}
    An indexed category $\mathbb{C}$ is $\mathcal{S}$-complete if it has $\mathcal{S}$-indexed products and $\mathcal{S}$-indexed finite limits (that is, each $C^I$ has finite limits preserved by the reindexing functors).
\end{definition}

\begin{definition}\label{def:indexed-continuous-functor}
    We call an indexed functor between $\mathcal{S}$-complete categories $F: \mathbb{C} \to \mathbb{D}$ $\mathcal{S}$-continuous if it is component-wise complete and the mate in the right is an isomorphism as well for every $x: I \to J$:

    \[\begin{tikzcd}[ampersand replacement=\&]
	{C^J} \&\& {C^I} \&\& {C^I} \&\& {D^I} \\
	\&\&\& \iff \\
	{D^J} \&\& {D^I} \&\& {C^J} \&\& {D^J}
	\arrow["{F^J}"', from=1-1, to=3-1]
	\arrow["{x^\ast}", from=1-1, to=1-3]
	\arrow["{x^\ast}"', from=3-1, to=3-3]
	\arrow["{F^I}", from=1-3, to=3-3]
	\arrow["\cong", shorten <=22pt, shorten >=22pt, Rightarrow, from=3-1, to=1-3]
	\arrow["{\Pi_x}"', from=1-5, to=3-5]
	\arrow["{F^I}"', from=3-5, to=3-7]
	\arrow["{F^I}", from=1-5, to=1-7]
	\arrow["{\Pi_x}", from=1-7, to=3-7]
	\arrow["\cong", shorten <=22pt, shorten >=22pt, Rightarrow, from=3-5, to=1-7]
\end{tikzcd}\]
\end{definition}

\section{Interlude: Beck-Chevalley}


Recall we can define adjointness using the $2$-cells, which are transformations $\alpha: F \implies G$ with, for every $x: J \to I$,

$${\alpha_J}_{x^\ast} \circ F^x = G^x \circ x^\ast \alpha_I$$

\noindent if we unpack what it means to be a modification in this context,

Thus we can define a notion of adjunction in this context, and if we do we'll end up with the following: $F \dashv G$ implies there we have componentise units and counits satisfying (recall the 2-cells are modifications)

\[\begin{tikzcd}
	{x^\ast} && {x^\ast G^I F^I} && {F^J x^\ast G^I} && {x^\ast F^I G^I} \\
	\\
	{G^JF^Jx^\ast} && {G^J x^\ast F^I} && {F^J G^J x^\ast} && {x^\ast}
	\arrow["{\eta^J_{x^\ast}}"', from=1-1, to=3-1]
	\arrow["{( G^JF^x )^{-1}}"', from=3-1, to=3-3]
	\arrow["{x^\ast \eta^I}", from=1-1, to=1-3]
	\arrow["{G^x F^I}", from=1-3, to=3-3]
	\arrow["{({F^x}_{G^J})^{-1}}", from=1-5, to=1-7]
	\arrow["{F^I G^x}"', from=1-5, to=3-5]
	\arrow["{\epsilon^I x^\ast}"', from=3-5, to=3-7]
	\arrow["{x^\ast \epsilon^J}", from=1-7, to=3-7]
\end{tikzcd}\]

Let us take the mate of $(F^x)^{-1}$:

\[\begin{tikzcd}
	{D^I} && {C^I} && {C^J} \\
	\\
	&& {D^I} && {D^J} && {C^J}
	\arrow["{x^\ast}", from=1-3, to=1-5]
	\arrow["{F^I}"', from=1-3, to=3-3]
	\arrow["{x^\ast}"', from=3-3, to=3-5]
	\arrow["{F^J}", from=1-5, to=3-5]
	\arrow["{(F^x)^{-1}}"', shorten <=22pt, shorten >=22pt, Rightarrow, from=1-5, to=3-3]
	\arrow[from=1-1, to=1-3]
	\arrow[""{name=0, anchor=center, inner sep=0}, no head, from=1-1, to=3-3]
	\arrow[from=3-5, to=3-7]
	\arrow[""{name=1, anchor=center, inner sep=0}, from=1-5, to=3-7]
	\arrow["{\epsilon^I}"', shorten <=8pt, shorten >=8pt, Rightarrow, from=1-3, to=0]
	\arrow["{\eta^J}", shorten <=8pt, shorten >=8pt, Rightarrow, from=1, to=3-5]
\end{tikzcd}\]

We get


\begin{align*}
         G^J x^\ast \epsilon^I \circ G^J (F^x)^{-1} G^I \circ \eta^J x^\ast G^I \\
         = G^J x^\ast \epsilon^I \circ G^x F^I G^I \circ x^\ast \eta^I G^I && \text{(by using the identities above)} \\
         = G^x \circ x^\ast G^I \epsilon^I \circ x^\ast \eta^I G^I && \text{(naturality of $G^x$)} \\
         = G^I  && \text{(functoriality and triangle identities)} \\
\end{align*}

Does converse hold? Yes, as long as $(F^x)^{-1}$ is invertible.


\begin{align*}
         G^x F^I \circ x^\ast \eta^I 
         \\
         = G^J x^\ast \epsilon^I_{F^I} \circ G^J (F^x_{F^I})^{-1}  \circ \eta^J_{x^\ast G^J} \circ x^\ast \eta^I && \text{($(F^x)^{-1}$ and $G^x$ are mates)}
         \\
         =  G^J x^\ast \epsilon^I_{F^I} \circ G^J (F^x_{F^I})^{-1} \circ G^J F^J x^\ast \eta^I \circ \eta^J_{x^\ast}  && \text{(naturality of $\eta^J$)}
         \\
         =  G^J x^\ast \epsilon^I_{F^I} \circ G^J x^\ast F^I \eta^I \circ G^J (F^x_{F^I})^{-1} \circ \eta^J_{x^\ast}  && \text{(naturality of $G^J (F^x)^{-1}$)}
         \\
         = G^J (F^x_{F^I})^{-1} \circ \eta^J_{x^\ast}  && \text{(functoriality and triangle identities)}
\end{align*}

Now, what does Beck-Chevalley has to do with all this?

Firstly, recall tha for

    \[\begin{tikzcd}[ampersand replacement=\&]
	I \&\& K \\
	\\
	J \&\& L
	\arrow["w", from=1-1, to=1-3]
	\arrow["x"', from=1-1, to=3-1]
	\arrow["z"', from=3-1, to=3-3]
	\arrow["y", from=1-3, to=3-3]
	\arrow["\lrcorner"{anchor=center, pos=0.125}, draw=none, from=1-1, to=3-3]
    \end{tikzcd}\]

    \noindent we have an isomorphism given by the $x^\ast$-transpose of

    $$x^\ast z^\ast \Pi_y A \xrightarrow[]{\cong} w^\ast y^\ast \Pi_y A \xrightarrow[]{w^\ast \epsilon^y_A} w^\ast A$$

    \noindent which is

    $$\Pi_x w^\ast \epsilon^y \circ \Pi_x {\phi^{-1}_{w, y} \phi_{x, z}}_{\Pi_y}  \circ \eta^x_{z^\ast \Pi_y}$$

    \noindent that is,

    \[\begin{tikzcd}
	{C^K} && {C^L} && {C^J} \\
	\\
	&& {C^K} && {C^I} && {C^J}
	\arrow["{z^\ast}", from=1-3, to=1-5]
	\arrow["{x^\ast}", from=1-5, to=3-5]
	\arrow["{y^\ast}"', from=1-3, to=3-3]
	\arrow["{w^\ast}"', from=3-3, to=3-5]
	\arrow["{\phi^{-1} \phi}"', shorten <=22pt, shorten >=22pt, Rightarrow, from=1-5, to=3-3]
	\arrow["{\Pi_x}"', from=3-5, to=3-7]
	\arrow[""{name=0, anchor=center, inner sep=0}, no head, from=1-5, to=3-7]
	\arrow[""{name=1, anchor=center, inner sep=0}, from=1-1, to=3-3]
	\arrow["{\Pi_y}", from=1-1, to=1-3]
	\arrow["\eta", shorten <=8pt, shorten >=6pt, Rightarrow, from=0, to=3-5]
	\arrow["\epsilon"', shorten <=8pt, shorten >=8pt, Rightarrow, from=1-3, to=1]
\end{tikzcd}\]

\noindent the mate of $\phi^{-1} \phi$.

\begin{definition}
    A comprehension category is a "cartesian" functor $Q$

    \[\begin{tikzcd}[ampersand replacement=\&]
	{\mathcal{E}} \&\& {\mathcal{S}^{\rightarrow}} \\
	\& {\mathcal{S}}
	\arrow["P"', from=1-1, to=2-2]
	\arrow["{\mathrm{cod}}", from=1-3, to=2-2]
	\arrow["Q", from=1-1, to=1-3]
\end{tikzcd}\]

    \noindent except that we don't assume $\mathcal{S}$ has pullbacks, hence $\mathrm{cod}$ is not necessarily a Grothendieck fibration, but $P$ is.
\end{definition}

\begin{definition}
    Given a comprehension category $Q: \mathcal{E} \to \mathcal{S}^\rightarrow$ and an $\mathcal{S}$-indexed category $\mathbb{C}$ given by $F: \mathcal{S}^{op} \to \mathrm{Cat}$, $[Q]_{\mathbb{C}}$ is the indexed functor given by

    \[\begin{tikzcd}[ampersand replacement=\&]
	{\mathcal{E}^{op}} \&\& {\mathcal{S}^{op}} \&\& {\mathrm{Cat}}
	\arrow[""{name=0, anchor=center, inner sep=0}, "{(\mathrm{cod} \circ Q)^{op}}", curve={height=-18pt}, from=1-1, to=1-3]
	\arrow[""{name=1, anchor=center, inner sep=0}, "{(\mathrm{dom} \circ Q)^{op}}"', curve={height=18pt}, from=1-1, to=1-3]
	\arrow["F", from=1-3, to=1-5]
	\arrow["{({\mathrm{arr} \ast Q})^{op}}"{description}, shorten <=5pt, shorten >=5pt, Rightarrow, from=0, to=1]
\end{tikzcd}\]
\end{definition}

Given a functor $P: \mathcal{E} \to \mathcal{S}$, we write $\mathrm{cart}{(P)}$ for the (possibly non-full) subcategory of $\mathcal{E}$ whose arrows are cartesian w.r.t. $P$. Consider thus the comprehension category $Q: \mathrm{cart}{(\mathrm{cod})} \to \mathcal{S}^\rightarrow$ and the induced indexed functor $F = [Q]_\mathbb{C}$.

\begin{theorem}
    The category $\mathbb{C}$ has $\mathcal{S}$-indexed coproducts iff $F$ has an indexed left adjoint.
\end{theorem}
\begin{proof}
    We've seen above that as long as the mate of each $G^x$ satisfies invertibility and we have componentwise adjoints, we can construct an indexed left adjoint.

    Let us look at the indexed functor $[Q]_\mathbb{C} = F: \mathbb{D} \to \mathbb{E}$. Let

    \[\begin{tikzcd}[ampersand replacement=\&]
	I \&\& K \\
	\\
	J \&\& L
	\arrow["w", from=1-1, to=1-3]
	\arrow["x"', from=1-1, to=3-1]
	\arrow["z"', from=3-1, to=3-3]
	\arrow["y", from=1-3, to=3-3]
	\arrow["\lrcorner"{anchor=center, pos=0.125}, draw=none, from=1-1, to=3-3]
    \end{tikzcd}\]
    
    \noindent be an arrow $x \to y$ in $\mathrm{cart}{(\mathrm{cod})}$.

    
    \begin{itemize}
    
        \item Its $x$-component is $x^\ast: C^J \to C^I$

        \item $F^(w, z)$ is given by $\phi_{w, y}^{-1} \phi_{x, z}$
    \end{itemize}

    It thus follows from what has been discussed previously.
\end{proof}

As a last thing, we are going to relate the Beck-Chevalley condition to spans, so that it'll be applied in the next section

\begin{remark}\label{rem:span-beck-chevalley}
    Let $\mathbb{D}$ be an $\mathcal{S}$-complete category. Consider the category $\langle{X, Y}$ of spans

    \[\begin{tikzcd}[ampersand replacement=\&]
	S \&\& Y \\
	\\
	X
	\arrow["x"', from=1-1, to=3-1]
	\arrow["y", from=1-1, to=1-3]
    \end{tikzcd}\]


    \noindent with fixed endpoints, and the obvious morphism.

    Then we can associate with the above span the functor $\Pi_y x^\ast: D^X \to D^Y$. In fact, we can functorially associate these by taking

    \[\begin{tikzcd}[ampersand replacement=\&]
	S \&\& Y \\
	\& {S^\prime} \\
	X
	\arrow["x"', from=1-1, to=3-1]
	\arrow["y", from=1-1, to=1-3]
	\arrow["u", from=1-1, to=2-2]
	\arrow["{x^\prime}", from=2-2, to=3-1]
	\arrow["{y^\prime}", from=2-2, to=1-3]
    \end{tikzcd}\]

    \noindent to 

    \[\begin{tikzcd}[ampersand replacement=\&]
	{\Pi_{y^\prime} {x^\prime}^\ast} \&\& {\Pi_{y^\prime} \Pi_u u^\ast {x^\prime}^\ast} \&\& {\Pi_y x^\ast}
	\arrow["{\Pi_{y^\prime} \eta^u {x^\prime}^\ast}", from=1-1, to=1-3]
	\arrow["{\psi_{y^\prime, u} \ast \phi_{x^\prime, u}}", from=1-3, to=1-5]
\end{tikzcd}\]

    \noindent thus yielding a contravariant action.

    Furthermore, given two spans in $<X, Y>, <Y, Z>$ respectively, we can form their composition in $<X, Z>$ by taking pullbacks

    \[\begin{tikzcd}[ampersand replacement=\&]
	U \&\& T \&\& Z \\
	\\
	S \&\& Y \\
	\\
	X
	\arrow["x"', from=3-1, to=5-1]
	\arrow["y"', from=3-1, to=3-3]
	\arrow["w", from=1-3, to=3-3]
	\arrow["z", from=1-3, to=1-5]
	\arrow["s"', from=1-1, to=3-1]
	\arrow["t", from=1-1, to=1-3]
	\arrow["\lrcorner"{anchor=center, pos=0.125}, draw=none, from=1-1, to=3-3]
    \end{tikzcd}\]

    \noindent provided such pullback exists. Then the Beck-Chevalley condition under the action above identifies the expression with $\Pi_z w^\ast \Pi_y x^\ast$ (that is, both paths are canonically isomorphic).
\end{remark}

We actually need something more precise, and the following result will be crucial.

\begin{lemma}\label{thm:beck-chevalley-fundamental-lemma}
    Consider the following situation:

    \[\begin{tikzcd}[ampersand replacement=\&]
	D \\
	\& C \& B \& Z \& D \&\&\& D \\
	\& A \& Y \&\&\& B \& Z \&\& A \& Y \\
	\& X \&\&\&\& Y \&\&\& X
	\arrow["b", from=2-2, to=2-3]
	\arrow["a"', from=2-2, to=3-2]
	\arrow["x"', from=3-2, to=4-2]
	\arrow["y"', from=3-2, to=3-3]
	\arrow["w", from=2-3, to=3-3]
	\arrow["z", from=2-3, to=2-4]
	\arrow["f", from=1-1, to=2-2]
	\arrow["u"', curve={height=12pt}, from=1-1, to=4-2]
	\arrow["v", curve={height=-18pt}, from=1-1, to=2-4]
	\arrow["d", from=2-2, to=3-3]
	\arrow["bf", from=2-5, to=3-6]
	\arrow["v", curve={height=-12pt}, from=2-5, to=3-7]
	\arrow["df"', curve={height=12pt}, from=2-5, to=4-6]
	\arrow["w"', from=3-6, to=4-6]
	\arrow["z", from=3-6, to=3-7]
	\arrow["af", from=2-8, to=3-9]
	\arrow["df", curve={height=-12pt}, from=2-8, to=3-10]
	\arrow["u"', curve={height=12pt}, from=2-8, to=4-9]
	\arrow["x"', from=3-9, to=4-9]
	\arrow["y", from=3-9, to=3-10]
\end{tikzcd}\]

Let the maps induced by the contravariant action above be denoted by $\gamma, \beta, \alpha$, respectively, and the Beck-Chevalley isomorphism by $\tau: w^\ast \Pi_y \to \Pi_b a^\ast$.

Then

$$\Pi_v \epsilon^{df} u^\ast \circ (\beta \ast \alpha) = \gamma\circ (\psi_{z, b} \ast \phi_{x, a}) \circ \tau $$
\end{lemma}
\begin{proof}
    We have

    $$\tau = \mathrm{Mate}(\phi_{y, a}^{-1} \phi_{w, b}) = \Pi_b a^\ast \epsilon^y \circ \Pi_b {\phi_{y, a}^{-1} \phi_{w, b}}_{\Pi_y} \circ \eta^b_{w^\ast \Pi_y}$$

    \begin{align*}
        \Pi_v \epsilon^{df} u^\ast \circ (\beta \ast \alpha)
        \\ = \Pi_v \epsilon^{df} u^\ast \circ ((\psi_{z, bf} \ast \phi_{w, bf}) \circ \Pi_z \eta^{bf}_{w^\ast}) \ast ((\psi_{y, af} \ast \phi_{x, af}) \circ \Pi_y \eta^{af}_{x^\ast}))
        \\ = \Pi_v \epsilon^{df}_{u^\ast} \circ ((\psi_{zb, f} \circ {\psi_{z, b}}_{\Pi_f} \ast \phi_{wb, f} \circ f^\ast \phi_{w, b}) \circ \Pi_z \Pi_b \eta^f_{b^\ast w^\ast} \circ \Pi_z \eta^b_{w^\ast})
        \\ \ast ((\psi_{ya, f} \circ {\psi_{y, a}}_{\Pi_f} \ast \phi_{xa, f} \circ f^\ast \phi_{x, a}) \circ \Pi_y \Pi_a \eta^f_{a^\ast x^\ast} \circ \Pi_y \eta^{a}_{x^\ast})) && \text{(1)}
        \\ = \Pi_v \epsilon^f_{u^\ast} \circ \Pi_v f^\ast \epsilon^d_{\Pi_f u^\ast} \circ ((\psi_{zb, f} \circ {\psi_{z, b}}_{\Pi_f} \ast f^\ast \phi_{w, b}) \circ \Pi_z \Pi_b \eta^f_{b^\ast w^\ast} \circ \Pi_z \eta^b_{w^\ast})
        \\ \ast (({\psi_{y, a}}_{\Pi_f} \ast \phi_{xa, f} \circ f^\ast \phi_{x, a}) \circ \Pi_y \Pi_a \eta^f_{a^\ast x^\ast} \circ \Pi_y \eta^{a}_{x^\ast})) && \text{(2)}
        \\ = \Pi_v \epsilon^f_{u^\ast} \circ \Pi_v f^\ast \epsilon^a_{\Pi_f u^\ast} \circ \Pi_v f^\ast a^\ast \epsilon^y_{\Pi_ a\Pi_f u^\ast} \circ (\psi_{zb, f} \circ {\psi_{z, b}}_{\Pi_f} \ast f^\ast \phi^{-1}_{y, a} \circ f^\ast \phi_{w, b}) \circ \Pi_z \Pi_b \eta^f_{b^\ast w^\ast} \circ \Pi_z \eta^b_{w^\ast})
        \\ \ast (\Pi_y \Pi_a \Pi_f (\phi_{xa, f} \circ f^\ast \phi_{x, a}) \circ \Pi_y \Pi_a \eta^f_{a^\ast x^\ast} \circ \Pi_y \eta^{a}_{x^\ast})) && \text{(3)}
        \\ = (\psi_{zb, f} \ast \phi_{xa, f}) \circ {\psi_{z, b}}_{\Pi_f f^\ast (xa)^\ast} \circ \Pi_z \Pi_b \Pi_f \epsilon^f_{f^\ast (xa)^\ast} \circ \Pi_z \Pi_b \Pi_f f^\ast \epsilon^a_{\Pi_f f^\ast (xa)^\ast} \circ \Pi_z \Pi_b \Pi_f f^\ast a^\ast \epsilon^y_{\Pi_ a\Pi_f f^\ast (xa)^\ast} \\
        \circ (\Pi_z \Pi_b \Pi_f (f^\ast \phi^{-1}_{y, a} \circ f^\ast \phi_{w, b})) \circ \Pi_z \Pi_b \eta^f_{b^\ast w^\ast} \circ \Pi_z \eta^b_{w^\ast})
        \\ \ast (\Pi_y \Pi_a \eta^f_{(xa)^\ast} \circ \Pi_y \Pi_a \phi_{x, a} \circ  \Pi_y \eta^{a}_{x^\ast})) && \text{(4)}
        \\ = (\psi_{zb, f} \ast \phi_{xa, f}) \circ {\psi_{z, b}}_{\Pi_f f^\ast (xa)^\ast} \circ \Pi_z \Pi_b \Pi_f f^\ast \epsilon^a_{(xa)^\ast} \circ \Pi_z \Pi_b \Pi_f f^\ast a^\ast \epsilon^y_{\Pi_ a (xa)^\ast}
        \\ \circ (\Pi_z \Pi_b \Pi_f (f^\ast \phi^{-1}_{y, a} \circ f^\ast \phi_{w, b})) \circ \Pi_z \Pi_b \eta^f_{b^\ast w^\ast} \circ \Pi_z \eta^b_{w^\ast}) \ast (\Pi_y \Pi_a \phi_{x, a} \circ  \Pi_y \eta^{a}_{x^\ast})) && \text{(5)}
        \\ = [ (\psi_{zb, f} \ast \phi_{xa, f}) \circ \Pi_{zb} \eta^f_{(xa)^\ast}] \circ {\psi_{z, b}}_{(xa)^\ast} \circ \Pi_z \Pi_b \epsilon^a_{(xa)^\ast} \circ \Pi_z \Pi_b a^\ast \epsilon^y_{\Pi_ a (xa)^\ast}
        \\ \circ (\Pi_z \Pi_b (\phi^{-1}_{y, a} \circ \phi_{w, b})) \circ \Pi_z \eta^b_{w^\ast}) \ast (\Pi_y \Pi_a \phi_{x, a} \circ  \Pi_y \eta^{a}_{x^\ast})) && \text{(6)}
        \\ = \gamma \circ {\psi_{z, b}}_{(xa)^\ast} \circ \Pi_z \Pi_b \epsilon^a_{(xa)^\ast} \circ \Pi_z \Pi_b a^\ast \epsilon^y_{\Pi_a (xa)^\ast} \circ \Pi_z \Pi_b a^\ast y^\ast (\Pi_y \Pi_a \phi_{x, a} \circ \Pi_y \eta^{a}_{x^\ast}) \circ \Pi_z \tau_{x^\ast}
        \\ = \gamma\circ (\psi_{z, b} \ast \phi_{x, a}) \circ \tau && \text{(7)}
    \end{align*}


\noindent
(1): Expansion of $\eta^{af}, \eta^{bf}$ and iso coherence \\
(2): Expression for $\epsilon^{df}$ \\
(3): Expression for $\epsilon^{ya}$ \\
(4): Naturality of the left epsilons and naturality of $\eta^f$ in right side \\
(5): Naturality of left epsilons and triangle identity for $f$ \\
(6): naturality of $\eta^f$ \\
(7): Naturality of the $\eta$s, triangle identity for $a$
\end{proof}

We shall prove a result that will come very handy to prove some comonad identities. 

First, we need a little technical lemma

\begin{lemma}
    The Beck-Chevalley morphism is compatible with gluing, that is

    \[\begin{tikzcd}[ampersand replacement=\&]
	A \&\& B \&\& E \&\& {[E]} \&\& {[B]} \&\& {[A]} \\
	\\
	C \&\& D \&\& F \&\& {[F]} \&\& {[D]} \&\& {[C]}
	\arrow["x", from=1-1, to=1-3]
	\arrow["y", from=1-3, to=1-5]
	\arrow["w"', from=1-1, to=3-1]
	\arrow["v"', from=3-1, to=3-3]
	\arrow["u"', from=3-3, to=3-5]
	\arrow["z", from=1-5, to=3-5]
	\arrow["h"{description}, from=1-3, to=3-3]
	\arrow["\lrcorner"{anchor=center, pos=0.125}, draw=none, from=1-1, to=3-3]
	\arrow["\lrcorner"{anchor=center, pos=0.125}, draw=none, from=1-3, to=3-5]
	\arrow["{x^\ast}", from=1-9, to=1-11]
	\arrow["{y^\ast}", from=1-7, to=1-9]
	\arrow[""{name=0, anchor=center, inner sep=0}, "{(yx)^\ast}", curve={height=-24pt}, from=1-7, to=1-11]
	\arrow["{u^\ast}"', from=3-7, to=3-9]
	\arrow["{v^\ast}"', from=3-9, to=3-11]
	\arrow[""{name=1, anchor=center, inner sep=0}, "{(uv)^\ast}"{description}, curve={height=24pt}, from=3-7, to=3-11]
	\arrow["{\Pi_z}"', from=1-7, to=3-7]
	\arrow["{\Pi_w}", from=1-11, to=3-11]
	\arrow["{\Pi_h}"{description}, from=1-9, to=3-9]
	\arrow[shorten <=22pt, shorten >=22pt, Rightarrow, from=3-7, to=1-9]
	\arrow[shorten <=22pt, shorten >=22pt, Rightarrow, from=3-9, to=1-11]
	\arrow[shorten <=2pt, Rightarrow, from=1, to=3-9]
	\arrow[shorten >=2pt, Rightarrow, from=1-9, to=0]
\end{tikzcd}\]
\end{lemma}
\begin{proof}
    Immediate from it being the described mate and the coherence of the composition isomorphisms
\end{proof}

We will not put out the whole definition of a bicategory, which the reader can get from \cite[Chapter 2]{Johnson2021}.

\begin{definition}
    Let $\mathcal{S}$ be a category with pullbacks. Then we have a bicategory $\mathrm{Span}(\mathcal{S})$ of spans with composition being "span composition" as previously defined, and horizontal composition being precisely taking the pullback-induced arrow. We still write $<X, Y>$ for the corresponding hom-categories of spans.
\end{definition}

We now refine our previous construction using bicategories.

\begin{lemma}\label{thm:span-pseudofunctorial-fundamental-lemma}
    Let $\mathbb{D}$ be an $\mathcal{S}$-complete category. We define a functor $\mathcal{G}$ from $\mathrm{Span}(\mathcal{S})$:

    \begin{itemize}
        \item $\mathcal{G}$ takes $X$ to $D^X$

        \item $\mathcal{G}$ takes a span

        \[\begin{tikzcd}[ampersand replacement=\&]
	S \&\& Y \\
	\\
	X
	\arrow["y", from=1-1, to=1-3]
	\arrow["x"', from=1-1, to=3-1]
        \end{tikzcd}\]

    \noindent to $\Pi_y x^\ast$

    \item $\mathcal{G}$ takes

    \[\begin{tikzcd}[ampersand replacement=\&]
	S \&\& Y \\
	\& T \\
	X
	\arrow["y", from=1-1, to=1-3]
	\arrow["x"', from=1-1, to=3-1]
	\arrow["z", from=2-2, to=3-1]
	\arrow["w"', from=2-2, to=1-3]
	\arrow["f"', from=1-1, to=2-2]
\end{tikzcd}\]

    \noindent to $(\psi_{w, f} \ast \psi_{z, f}) \circ \Pi_w \eta^f_{z^\ast}$.
    \end{itemize}

    Let us write $D^\square$ for the $2$-category with objects $D^X$ whose hom-categories are inverted, that is, $[D^X, D^Y]^{op}$. Then $\mathcal{G}$ defines a pseudofunctor $\mathrm{Span}{(\mathcal{S})} \to D^\square$.
\end{lemma}
\begin{proof}
    The coherence isos for composition are given by Beck-Chevalley and the coherence isos for $\mathbb{D}$. The associativity commutativity conditions follow from the previous lemma.

    For instance, let us consider the condition

        \[\begin{tikzcd}
	& {<X, Y> \times <Y, Z>} \\
	\\
	{<X, Z>} && {[D^X, D^Y]^{op} \times [D^Y, D^Z]^{op}} \\
	& {[D^X, D^Z]^{op}}
	\arrow["{\mathcal{G} \times \mathcal{G}}", from=1-2, to=3-3]
	\arrow["\times"', from=1-2, to=3-1]
	\arrow["\ast", from=3-3, to=4-2]
	\arrow["{ \xi_{T, S}^{-1} \mathcal{G} \xi_{V, U}}"', from=3-1, to=4-2]
    \end{tikzcd}\]

\noindent where $\xi_{S, T}: \mathcal{G}(S) \circ \mathcal{T} \to \mathcal{G}(S \circ T)$ is the coherence iso given by the obvious composition of the Beck-Chevalley isomorphism and $\phi, \psi$.

Let us consider the above situation with the spans

\[\begin{tikzcd}
	\textcolor{rgb,255:red,153;green,153;blue,153}{T \circ S} && T && Z \\
	& \textcolor{rgb,255:red,153;green,153;blue,153}{V \circ U} && V \\
	S && Y \\
	& U \\
	X
	\arrow["y", from=3-1, to=3-3]
	\arrow["w"', from=1-3, to=3-3]
	\arrow["v", from=1-3, to=2-4]
	\arrow[from=2-4, to=3-3]
	\arrow["z", from=1-3, to=1-5]
	\arrow[from=2-4, to=1-5]
	\arrow[from=4-2, to=3-3]
	\arrow[from=4-2, to=5-1]
	\arrow["x"', from=3-1, to=5-1]
	\arrow["u"', from=3-1, to=4-2]
	\arrow[color={rgb,255:red,153;green,153;blue,153}, from=2-2, to=4-2]
	\arrow["{u \times v}", color={rgb,255:red,153;green,153;blue,153}, from=1-1, to=2-2]
	\arrow[color={rgb,255:red,153;green,153;blue,153}, from=2-2, to=2-4]
	\arrow[color={rgb,255:red,153;green,153;blue,153}, from=1-1, to=3-1]
	\arrow[color={rgb,255:red,153;green,153;blue,153}, from=1-1, to=1-3]
\end{tikzcd}\]

Now, we have

\begin{align*}
    \Pi_{z p_T} \epsilon^{d_{S, T}}_{(x p_S)^\ast} \circ (\mathcal{G}(p_T) \ast \mathcal{G}(p_S))
    \\ = \Pi_{z p_T} \epsilon^{d_{S, T}}_{(x p_S)^\ast} \circ (((\psi_{z, p_T} \ast \phi_{w, P_T}) \circ \Pi_z \eta^{p_T}_{w^\ast}) \ast ((\psi_{y, p_S} \ast \phi_{x, p_S}) \circ \Pi_y \eta^{p_S}_{x^\ast}))
    \\ = \Pi_{z p_T} \epsilon^{p_S}_{(x p_S)^\ast} \circ \Pi_{z p_T} p_S^\ast \epsilon^{y}_{\Pi_{p_S} (x p_S)^\ast} \circ \Pi_{z p_T} (\phi_{y, p_S}^{-1} \ast \psi_{x, p_S}^{-1})_{(x p_S)^\ast} \circ (((\psi_{z, p_T} \ast \phi_{w, P_T}) \circ \Pi_z \eta^{p_T}_{w^\ast}) \ast ((\psi_{y, p_S} \ast \phi_{x, p_S}) \circ \Pi_y \eta^{p_S}_{x^\ast}))
    \\ = \psi_{z, p_T} \circ \Pi_{z} \Pi_{p_T} \epsilon^{p_S}_{(x p_S)^\ast} \circ \Pi_{z} \Pi_{p_T} p_S^\ast \epsilon^{y}_{\Pi_{p_S} (x p_S)^\ast} \circ ((\Pi_{z} \Pi_{p_T} (\phi^{-1}_{y, P_S} \phi_{w, P_T}) \circ \Pi_z \eta^{p_T}_{w^\ast}) \ast (\Pi_{y} \Pi_{P_S} (\phi_{x, p_S}) \circ \Pi_y \eta^{p_S}_{x^\ast})) && \text{(1)}
    \\ = (\psi_{z, p_T} \ast \phi_{x, P_s}) \circ \Pi_{z} \Pi_{p_T} p_S^\ast \epsilon^{y}_{\Pi_{p_S} (x p_S)^\ast} \circ (\Pi_{z} \Pi_{p_T} (\phi^{-1}_{y, P_S} \phi_{w, P_T}) \circ \Pi_z \eta^{p_T}_{w^\ast}) && \text{(2)}
    \\ = \xi_{S, T}
\end{align*}

\noindent
(1): Naturality of $\psi_{z, p_T}$, cancelled $\psi_{z, p_S}$ \\
(2): Naturality of the epsilons, triangle identity for $p_S$

Now it suffices to invoke Lemma \ref{thm:beck-chevalley-fundamental-lemma} and we're done.

Also, note that we have the expected identities for different pullbacks in Beck-Chevalley, that is, if we have weo candidate pullbacks

\[\begin{tikzcd}[ampersand replacement=\&]
	\& D \\
	E \&\&\& B \\
	\\
	\& A \&\& C
	\arrow["a", from=1-2, to=4-2]
	\arrow["b", from=1-2, to=2-4]
	\arrow["{a^\prime}"', from=2-1, to=4-2]
	\arrow["{b^\prime}", from=2-1, to=2-4]
	\arrow["u"', from=4-2, to=4-4]
	\arrow["v", from=2-4, to=4-4]
	\arrow["\cong"', draw=none, from=1-2, to=2-1]
	\arrow["\alpha"{description}, from=1-2, to=2-1]
\end{tikzcd}\]

\noindent then (where $\tau, \tau^\prime$ are the B-C isos for $D ,E$, respectively),

\begin{align*}
    \mathcal{G}(\alpha) \circ \tau^\prime = (\psi_{b^\prime, \alpha} \ast \phi_{a^\prime, \alpha}) \circ \Pi_{b^\prime} \eta^{\alpha}_{{a^\prime}^\ast} \circ \tau^\prime
    \\ = (\psi_{b^\prime, \alpha} \ast \phi_{a^\prime, \alpha}) \circ \Pi_{b^\prime} \eta^{\alpha}_{{a^\prime}^\ast} \circ \Pi_{b^\prime}{a^\prime}^{\ast} \epsilon^u \circ \Pi_{b^\prime} {\phi^{-1}_{u, a^\prime} \phi_{v, b^\prime}}_{\Pi_u} \circ \eta^{b^\prime}_{v^\ast \Pi_u}
    \\ = \Pi_{b}{a}^{\ast} \epsilon^u \circ \Pi_b {\phi_{a^\prime, \alpha}}_{u^\ast \Pi_u} \circ \Pi_{b} \alpha^\ast {\phi^{-1}_{u, a^\prime} \phi_{v, b^\prime}}_{\Pi_u} \circ {\psi_{b^\prime, \alpha}}_{\alpha^\ast {b^\prime}^\ast v^\ast \Pi_{u}} \circ \Pi_{b^\prime} \eta^\alpha_{{b^\prime}^\prime v^\ast \Pi_{u}} \circ \eta^{b^\prime}_{v^\ast \Pi_u} && \text{(1)}
    \\ = \Pi_{b}{a}^{\ast} \epsilon^u \circ \circ \Pi_{b} \alpha^\ast {\phi^{-1}_{u, a} \phi_{v, b}}_{\Pi_u} \circ \Pi_b {\phi_{b^\prime, \alpha}}_{v^\ast \Pi_u} \circ {\psi_{b^\prime, \alpha}}_{\alpha^\ast {b^\prime}^\ast v^\ast \Pi_{u}} \circ \Pi_{b^\prime} \eta^\alpha_{{b^\prime}^\prime v^\ast \Pi_{u}} \circ \eta^{b^\prime}_{v^\ast \Pi_u} && \text{(2)}
    \\ = \tau && \text{(3)}
\end{align*}

\noindent
(1): Naturality of $\eta^\alpha, \psi_{\alpha, b^\prime}, \phi_{a^\prime, \alpha}$ \\
(2): Associativity condition for isos $\phi$ \\
(3): Identity for $\eta^{yx}$

\end{proof}

\section{(Co) monadicity of diagrams}
\begin{definition}
    Let $\mathbb{E}$ be a $\mathcal{S}$-indexed category. A $\mathbb{C}$-shaped diagram in $\mathbb{E}$ consists of the following data:

    \begin{itemize}
        \item An object $P \in \mathbb{E}^{C_0}$
        \item An arrow $\Phi: d_0^\ast P \to d_1^\ast P$
    \end{itemize}

    such that $i^\ast \Phi = 1_P$ and $c^\ast \Phi = p_1^\ast \Phi \circ p_0^\ast \Phi$ modulo the coherence isos, that is,

    \begin{itemize}
        \item $ \phi_{d_0, i} i^\ast \Phi \phi_{d_1, i}^{-1} = 1_{1^\ast P}$

        \item $c^\ast \Phi = \phi_{d_1, c}^{-1} \phi_{d_1, p_1} p_1^\ast \Phi \phi_{d_0, p_1}^{-1} \phi_{d_1, p_0} p_0^\ast \Phi \phi_{d_0, p_0}^{-1} \phi_{d_0, c}$
    \end{itemize}
    
    Morphisms between such diagrams are defined in the obvious way.

    We write $\mathbb{E}^\mathbb{C}$ for the corresponding category.
\end{definition}

It is not hard to see that under externalization of small categories we have an equivalence of indexed functors $\underline{\mathbb{C}} \to \mathbb{E}$ and $\mathbb{C}$-shaped diagrams in $\mathbb{E}$. A nice intuition for diagrams is considering the notion of in-universe functor as an action in the slice indexing of $\mathcal{S}$ over itself. In fact, the reader might check that under the standard $\mathrm{Set}$-indexing of categories, diagrams correspond to the usual notion, yielding a mapping whose functoriality is expressed by the last two conditions. Internal diagrams give rise to diagram categories, with $\left( \mathbb{E}^\mathbb{C} \right)^I = \mathbb{E}^{\mathbb{C} \times I}$, with the obvious transition functors.

Now, if we wish to talk about the indexed version of the limit and colimit functors, it seems reasonable to look for adjoints of the pre-composition indexed functor $F^\ast: \mathbb{E}^\mathbb{D} \to \mathbb{E}^\mathbb{C}$ (where $F$ is an internal functor). In fact, we look at adjoints of the indexing functors, by endowing $\mathbb{E}$ with a $\mathrm{Cat}(\mathcal{S})$-indexed structure in the obvious way.

\begin{theorem}
    Let $\mathbb{E}$ have the canonical $\mathrm{Cat}(\mathcal{S})$-indexed category structure. Then if $\mathbb{E}$ is (co) complete (in the indexed sense), its transition functors have right (left) adjoints. Moreover, $\mathbb{D}^\mathbb{C}$ has $\mathcal{S}$-indexed (co)products.
\end{theorem}

First, we need to prove

\begin{theorem}
    Let $\mathbb{D}$ have $\mathcal{S}$-indexed products. Then the category $\mathcal{D}^\mathbb{C}$ (we use this notation for the non-indexed version) is comonadic over $\mathcal{D}^\mathbb{C}$.
\end{theorem}
\begin{proof}
 Thanks to the results laid out in the previous section, we shall be able to easily prove this result.

   Consider the following map

    \[\begin{tikzcd}[ampersand replacement=\&]
	{\Pi_{d_0} d_1^\ast} \&\& {\Pi_{d_0} \Pi_{i} i^\ast d_1^\ast} \&\& 1
	\arrow["\cong", from=1-3, to=1-5]
	\arrow["{\psi_{d_0, i} \ast \phi_{d_1, i}}"', draw=none, from=1-3, to=1-5]
	\arrow["{\Pi_{d_0} \eta_{d_1^\ast}}"', from=1-1, to=1-3]
    \end{tikzcd}\]

\noindent which we denote by $\eta$.

Now, if $(P, \Phi)$ is an internal diagram, consider the $d_0^\ast$-transpose $\Phi^\prime: P \to \Pi_{d_0} d_1^\ast P$.

Then we have the unit property

\begin{align*}
    ({\theta^\prime}^{-1} \psi_{d_1, i} \ast \theta^{-1} \phi_{d_0, i}) \circ \Pi_{d_0} \eta^i_{d_1^\ast} \circ \Pi_{d_0} \Phi \circ \eta^{d_0}_P \\
    =  ({\theta^\prime}^{-1} \psi_{d_1, i} \ast \theta^{-1} \phi_{d_0, i}) \circ \circ \Pi_{d_0} \Pi_i i^\ast \Phi \circ \Pi_{d_0} \eta^i_{d_0^\ast} \circ \eta^{d_0}_P  && \text{(naturality of $\eta^i$)} \\
    = 1 && \text{(unit identities proved in Remark \ref{rem:unit-identities})}
\end{align*}

Note that by the selfsame identities (that is, right and left side are isos) and $id_0 = 1$, we can recover such equality so converse holds.

Consider also

\[\begin{tikzcd}[ampersand replacement=\&]
	{\Pi_{d_0} d_1^\ast} \&\& {\Pi_{d_0} \Pi_{c} c^\ast d_1^\ast} \&\& {\Pi_{d_0} \Pi_{p_0} p_1^\ast d_1^\ast} \&\& {\Pi_{d_0} d_1^\ast \Pi_{d_0} d_1^\ast}
	\arrow["{\Pi_{d_0} \eta_{d_1^\ast}}", from=1-1, to=1-3]
	\arrow["\cong", from=1-3, to=1-5]
	\arrow[from=1-5, to=1-7]
\end{tikzcd}\]

\noindent which we denote by $\mu$, where the last arrow (which we denote by $\alpha$) is induced by the Beck-Chevalley condition on the pullback

\[\begin{tikzcd}[ampersand replacement=\&]
	{C_2} \&\& {C_1} \\
	\\
	{C_1} \&\& {C_0}
	\arrow["{p_1}", from=1-1, to=1-3]
	\arrow["{p_0}"', from=1-1, to=3-1]
	\arrow["{d_1}"', from=3-1, to=3-3]
	\arrow["{d_0}", from=1-3, to=3-3]
	\arrow["\lrcorner"{anchor=center, pos=0.125}, draw=none, from=1-1, to=3-3]
\end{tikzcd}\]

Then we have the composition property for the transpose of our diagram $\Phi$:

\begin{align*}
    \Pi_{d_0} \alpha^{-1}_{d_1^\ast} \circ (\psi_{d_0, p_0}^{-1} \psi_{d_0, c} \ast \phi_{d_1, p_1}^{-1} \phi_{d_1, c}) \circ \Pi_{d_0} \eta^c_{d_1^\ast} \circ \Pi_{d_0} \Phi \circ \eta^{d_0}
    \\ = \Pi_{d_0} d_1^\ast \Pi_{d_0} \Phi \circ \Pi_{d_0} \alpha^{-1}_{d_0^\ast} \circ \Pi_{d_0} \Pi_{p_0} \left(\phi_{d_0, p_1}^{-1} \circ \phi_{d_1, p_0} \circ p_0^\ast \Phi \circ \phi_{d_0, p_0}^{-1} \circ \phi_{d_0, c} \right) \circ (\psi_{d_0, p_0}^{-1} \psi_{d_0, c})_{c^\ast d_0^\ast} \circ \Pi_{d_0} \eta^c_{d_0^\ast} \circ \eta^{d_0} && \text{(1)}
    \\ = \Pi_{d_0} d_1^\ast \Pi_{d_0} \Phi \circ \Pi_{d_0} \alpha^{-1}_{d_0^\ast} \circ \Pi_{d_0} \Pi_{p_0} \left(\phi_{d_0, p_1}^{-1} \circ \phi_{d_1, p_0} \circ p_0^\ast \Phi \right) \circ \Pi_{d_0} \eta^{p_0}_{d_0^\ast} \circ \eta^{d_0} && \text{(2)}
    \\ = \Pi_{d_0} d_1^\ast \Pi_{d_0} \Phi \circ \Pi_{d_0} \alpha^{-1}_{d_0^\ast} \circ \Pi_{d_0} \Pi_{p_0} p_1^\ast (\epsilon^{d_0}_{d_0^\ast} \circ d_0^\ast \eta^{d_0}) \circ \Pi_{d_0} \Pi_{p_0} \left(\phi_{d_0, p_1}^{-1} \circ \phi_{d_1, p_0} \right) \circ \Pi_{d_0} \eta^{p_0}_{d_1^\ast} \circ \Pi_{d_0} \Phi \circ \eta^{d_0} && \text{(3)}
    \\ = \Pi_{d_0} d_1^\ast \Pi_{d_0} \Phi \circ \Pi_{d_0} d_1^\ast \eta^{d_0} \circ \Pi_{d_0} \Phi \circ \eta^{d_0} && \text{(4)}
\end{align*}

\noindent
(1): Naturality of $\eta^c$, expansion of $c^\ast \Phi$, naturality of $\psi, \alpha^{-1}$ \\
(2): Identity for $\eta^{yx}$ for $d_0c, p_0c$ \\
(3): Swap $\eta^{p_0}$ and $\Phi$, then apply the triangle identity in reverse for $d_0$ \\
(4): Pass new $\eta^{d_0}$ rightward using naturality, then we get the exact expression for $\alpha$ \\

Note for converse, we use step 0 to 1, cutting off all isos, and noting that the expression we're left is nothing more than two transposes.

Now we verify that $(\Pi_{d_0} d_1^\ast, \eta, \mu)$ has comonad structure. Recall Remark \ref{rem:span-beck-chevalley}.

In the language of spans, an internal category is given by spans

\[\begin{tikzcd}[ampersand replacement=\&]
	{C_1} \& {C_0} \& {C_2} \& {C_1} \& {C_0} \& {C_0} \\
	{C_0} \&\& {C_1} \&\& {C_0}
	\arrow["{d_1}"', from=1-1, to=2-1]
	\arrow["{d_0}", from=1-1, to=1-2]
	\arrow["{p_1}"', from=1-3, to=2-3]
	\arrow["{p_0}", from=1-3, to=1-4]
	\arrow[no head, from=1-5, to=2-5]
	\arrow[no head, from=1-5, to=1-6]
\end{tikzcd}\]

\noindent (where the second is the obvious composite) and the appropriate arrows $c, i$ between. Thanks to Lemma \ref{thm:span-pseudofunctorial-fundamental-lemma}, we can easily prove commutativity as desired. Recall the notation with the functor $\mathcal{G}$ and transition isos $\xi_{T, S}$. Then for $c(1 \times i)$, for instance, it follows immediately that

$$1 = \mathcal{G}(c (1 \times i)) = \mathcal{G}(1 \times i) \circ \mathcal{G}(c) = \xi_{C_1, C_0}^{-1} (\mathcal{G}(i) \ast) \xi_{C_1, C_1}^{-1} \mathcal{G}(c) = \ \eta^\prime \ast \Pi_{d_0} d_1^\ast \circ \mu^\prime$$

\noindent by choosing $C_1 \circ C_0 = C_1$ (noting that the Beck-Chevalley coherence isos are entirely compatible with the pullback isomorhisms, as proved in the end of the previous section).

Also,

$$\mathcal{G}(c(1 \times c)) = \xi_{C_2, C_1} (\mathcal{G} \ast \mathcal{G}(1) \xi_{C_1, C_1}^{-1} \mathcal{G}(c)$$

Now note by coherence

\[\begin{tikzcd}[ampersand replacement=\&]
	{\mathcal{G}(C_2) \circ \mathcal{G}(C_1)} \&\& {\mathcal{G} (C_1) \circ \mathcal{G} (C_1) \circ \mathcal{G} (C_1)} \\
	\\
	{\mathcal{G}(C_3)} \&\& {\mathcal{G}(C_1) \circ \mathcal{G}(C_2)}
	\arrow["{\xi_{C_2, C_1}}"', from=1-1, to=3-1]
	\arrow["{{\xi_{C_1, C_1}}_{\Pi_{d_0} d_1^\ast}^{-1}}", from=1-1, to=1-3]
	\arrow["{\Pi_{d_0} d_1^\ast \xi_{C_1, C_1}^{-1}}"', from=3-3, to=1-3]
	\arrow["{\xi_{C_2, C_1}^{-1}}"', from=3-1, to=3-3]
\end{tikzcd}\]

\noindent so postcomposing we get the desired equality.
\end{proof}


We've just proved a crucial theorem towards our goal, but we still need another to constructed the desired colimits, which we'll prove in the next section

\section{Interlude: morphisms of monads}

Just like in the previous section, we lay out results which'll be useful later on the next section. Let $T, S$ be two monads. We write $\eta^T, \eta^S, \mu^T, \mu^S$ for the respective unit/multiplication arrows. Let

\[\begin{tikzcd}[ampersand replacement=\&]
	{C^T} \&\& {D^S} \\
	\\
	C \&\& D
	\arrow["Q"', from=1-1, to=1-3]
	\arrow["R"', from=3-1, to=3-3]
	\arrow["{U^T}", shift left, from=1-1, to=3-1]
	\arrow["{U^S}", shift left, from=1-3, to=3-3]
	\arrow["{F^T}", shift left, from=3-1, to=1-1]
	\arrow["{F^S}", shift left, from=3-3, to=1-3]
\end{tikzcd}\]

\noindent be a square s.t. $R U^T = U^S Q$.

Let

$$\delta = \mathrm{Mate}(R U^T = U^S Q) = \epsilon^S_{Q F^T} \circ F^S R \eta^T$$

\noindent and $\lambda = U^S \delta$. Then clearly by the triangle inequality

\[\begin{tikzcd}[ampersand replacement=\&]
	R \&\& SR \\
	\\
	\&\& RT
	\arrow["{R \eta^T}"', from=1-1, to=3-3]
	\arrow["\lambda", from=1-3, to=3-3]
	\arrow["{\eta^S_R}", from=1-1, to=1-3]
\end{tikzcd}\]

Furthermore,

\begin{align}
    R \mu^T \circ \lambda_T \circ S \lambda
    \\ = U^S Q \epsilon^T_{F^T} \circ (U^S \epsilon^S_{Q F^T T} \circ S R \eta^T_T) \circ (S U^S \epsilon^S_{Q F^T} \circ S S R \eta^T)
    \\ = U^S \epsilon^S_{Q F^T} \circ (S U^S \epsilon^S_{Q F^T} \circ S S R \eta^T) && \text{(Naturality of $\epsilon^S$ and triangle identity)}
    \\ = (U^S \epsilon^S_{Q F^T} \circ S R \eta^T) \circ U^S \epsilon^S_{F^S R} && \text{(Naturality of the left $\epsilon^S$)}
    \\ \lambda \circ \mu^R
\end{align}

\noindent so we have

\[\begin{tikzcd}[ampersand replacement=\&]
	{S^2 R} \&\&\& SR \\
	SRT \\
	{RT^2} \&\&\& RT
	\arrow["{\mu^S_R}", from=1-1, to=1-4]
	\arrow["\lambda", from=1-4, to=3-4]
	\arrow["{S \lambda}"', from=1-1, to=2-1]
	\arrow["{\lambda_T}"', from=2-1, to=3-1]
	\arrow["{R \mu^T}"', from=3-1, to=3-4]
\end{tikzcd}\]

In fact, the reader might know that monads correspond to lax functors $1 \to \mathrm{Cat}$, and in fact the transformation $\lambda$ yields a lax transformation between such functors.

\begin{lemma}
    We have $Q(A, a) = (RA, Ra \circ \lambda_A)$
\end{lemma}
\begin{proof}
    $$R U^T \epsilon^T_{(A, a)} \circ \lambda_{U_T} = U^S \epsilon^S_{Q}$$

    \noindent by naturality of $\epsilon^S$ and the triangle identity.
\end{proof}

Thus there is an equivalence between the data of $(R, \lambda)$ and the data given by the square at the start of the section, as we can reconstruct $Q: C^S \to D^T$ from $R$. We call the pair $(R, \lambda)$ satisfying the aforementioned identities a morphism of monads. In fact, we can compose these 1-cells. Let $(R, \lambda): T \to S,  (L, \delta): S \to K$ be given. Then the obvious composition is the desired composition: $L \lambda \circ \delta_R$.

Of course, if one unravels the definition of modification (2-cells for lax functors) then we have that a 2-cell $(R, \lambda) \to (L, \delta)$ consists of $\gamma: R \implies L$ s.t. $\gamma_T \circ \lambda = \delta \circ S \gamma$. Furthermore, we have horizontal composition by taking the horizontal composition of the underlying transformations.

We define the notion of comorphism of monads $(L, \delta): S \to T$ as being induced by an oplax transformation instead, which gives us the dual identities:

\[\begin{tikzcd}[ampersand replacement=\&]
	L \&\& LS \\
	\\
	\&\& TL
	\arrow["{L \eta^S}", from=1-1, to=1-3]
	\arrow["\delta", from=1-3, to=3-3]
	\arrow["{\eta^T_L}"', from=1-1, to=3-3]
\end{tikzcd}\]

\[\begin{tikzcd}[ampersand replacement=\&]
	{LS^2} \&\&\& LS \\
	TLS \\
	{T^2 L} \&\&\& TL
	\arrow["{L \mu^S}", from=1-1, to=1-4]
	\arrow["\delta", from=1-4, to=3-4]
	\arrow["{\delta_S}"', from=1-1, to=2-1]
	\arrow["{T \delta}"', from=2-1, to=3-1]
	\arrow["{\mu^T_L}"', from=3-1, to=3-4]
\end{tikzcd}\]

We shall write $T \rightharpoonup S$ and $T \rightharpoondown S$ for morphisms and comorphisms of moands respectively, from now own.

\begin{lemma}
    Suppose $(R, \lambda): T \rightharpoonup S$ and $R$ has a left adjoint $L$. Then the mate of $\lambda$ yields $(L, \delta): S \rightharpoondown T$.
\end{lemma}
\begin{proof}
    Recall that by mate we mean

    $$\delta = \epsilon^L_{TL} \circ L \lambda_L \circ L S \eta^L$$
    
    Verification of the unit identity is straightforward. To verify the composition identity:

    \begin{align*}
        \mu^T_L \circ T \delta \circ \delta_S
        \\ = \epsilon^S_{TL} \circ LR \mu^T \circ L \lambda_{TLS} \circ LSR \delta \circ LS \eta^L_S && \text{(Naturality of $\epsilon^S$ and $\lambda$)}
        \\ = \epsilon^S_{TL} \circ LR \mu^T_L \circ L \lambda_{TLS} \circ LSR (\epsilon^L_{TL} \circ L \lambda_{L} \circ LS \eta^L) \circ LS \eta^L && \text{(Expanded expression)}
        \\ = \epsilon^S_{TL} \circ LR \mu^T_L \circ L \lambda_{TL} \circ LS \lambda_{L} \circ L S S \eta^L && \text{(Naturality of rightmost $\eta^L$ and triangle identity)}
        \\ = \epsilon^S_{TL} \circ L (\lambda \circ \mu^S_R)_L \circ L S S \eta^L  && \text{(Composition identity)}
        \\ = \delta \circ L \mu^S
    \end{align*}
\end{proof}

Just like for morphisms of monads, if we are given

\[\begin{tikzcd}[ampersand replacement=\&]
	{C^T} \&\& {D^S} \\
	\\
	C \&\& D
	\arrow["{U^T}", shift left, from=1-1, to=3-1]
	\arrow["{U^S}", shift left, from=1-3, to=3-3]
	\arrow["{F^T}", shift left, from=3-1, to=1-1]
	\arrow["{F^S}", shift left, from=3-3, to=1-3]
	\arrow["P"', from=1-3, to=1-1]
	\arrow["L"', from=3-3, to=3-1]
\end{tikzcd}\]

\noindent s.t. $F^T L = P F^S$, we can get a comorphism $(L, \delta): S \rightharpoondown T$ by taking the mate $\pi$ of the equality

\[\begin{tikzcd}[ampersand replacement=\&]
	D \&\& C \\
	\\
	{D^S} \&\& {C^T}
	\arrow["Q", from=1-1, to=1-3]
	\arrow["{F^T}", from=1-3, to=3-3]
	\arrow["P"', from=3-1, to=3-3]
	\arrow["{F^S}"', from=1-1, to=3-1]
	\arrow["1"', shorten <=22pt, shorten >=22pt, Rightarrow, from=1-3, to=3-1]
\end{tikzcd}\]

\noindent then setting $\delta = \pi_{F^S}$. We leave it to the interested reader to verify the identities are satisfied.

\begin{lemma}
    Let

    \[\begin{tikzcd}[ampersand replacement=\&]
	{C^T} \&\& {D^S} \\
	\\
	C \&\& D
	\arrow["Q"', shift right, from=1-1, to=1-3]
	\arrow["R"', shift right, from=3-1, to=3-3]
	\arrow["{U^T}", shift left, from=1-1, to=3-1]
	\arrow["{U^S}", shift left, from=1-3, to=3-3]
	\arrow["{F^T}", shift left, from=3-1, to=1-1]
	\arrow["{F^S}", shift left, from=3-3, to=1-3]
	\arrow["P"', shift right, from=1-3, to=1-1]
	\arrow["L"', shift right, from=3-3, to=3-1]
\end{tikzcd}\]

\noindent such that

\[\begin{tikzcd}[ampersand replacement=\&]
	{(R U^T, \_)} \&\& {(\_, F^T L)} \\
	\\
	{(U^S Q, \_)} \&\& {(\_, P F^S)}
	\arrow["{=}"', no head, from=1-1, to=3-1]
	\arrow["{=}", no head, from=1-3, to=3-3]
	\arrow["\cong"{description}, from=3-1, to=3-3]
	\arrow["\cong"{description}, from=1-1, to=1-3]
\end{tikzcd}\]

Then the two ways of obtaining a comorphism $(L, \delta): S \rightharpoondown T$ give the same arrow.
\end{lemma}
\begin{proof}

We talk about horizontal and vertical mates, in the obvious sense. Note that by the condition on the transposes, the identity transformation representing $P F^S = F^T L$ is given by the horizontal transpose followed by the vertical transpose of $R U^T = U^S Q$. Now, if we write $\circ_V, \circ_H$ for vertical and horizontal $2$-cell square composition, respectively, we have:

\begin{align*}
\mathrm{VMate}( (R U^T = U^S Q) \circ_V  \mathrm{HMate}(R U^T = U^S Q))
\\ = \mathrm{VMate}(R U^T = U^S Q) \circ_V (R U^T = U^S Q)
\end{align*}
    
\end{proof}

We thus obtain the following results which are crucial for the next section in terms of motivation:

\begin{corollary}\label{thm:comorphism-monads-corollary}
    With the same conditions as the previous lemma, one has

    $$P \epsilon^{F^S}_{F^S} = \epsilon^{F^T}_{F^T L} \circ F^T \delta = \epsilon^{F^T}_{F^T L} \circ F^T \epsilon^L_{TL} \circ F^T L U^S \epsilon^{F^s}_{L} \circ F^T L S R \eta^T_L \circ F^T L S \eta^L$$
\end{corollary}

\begin{lemma}
    If we write

    $$\pi = \mathrm{Mate}(R U^T = U^S Q) = \epsilon^S_{Q F^T} \circ F^S R \eta^T$$

    \noindent we have
    
    $$\pi_L \circ F^S \eta^S = \eta^Q_{F^S}$$
\end{lemma}
\begin{proof}
    \begin{align*}
        \pi_L \circ F^S \eta^S
        \\ = \epsilon^S_{Q F^T L} \circ F^S R \eta^T_L \circ F^S \eta^R
        \\ = \epsilon^S_{Q F^T L} \circ F^S U^S \eta^Q_{F^S} \circ F^S \eta^S && \text{(Commutativity condition on transposes)}
        \\ = \eta^Q_{F^S}
    \end{align*}
\end{proof}

Before we move on, we still need the following result crucially:

\begin{lemma}
    Let

    $$\pi = \mathrm{Mate}(R U^T = U^S Q) = \epsilon^S_{Q F^T} \circ F^S R \eta^T$$

    $$\varphi = \pi_L \circ F^S \eta^L$$

    $$\omega = \epsilon^{F^T}_{F^T L} \circ F^T \delta$$

    Without any assumption on the existence of a left adjoint for $Q$, we have:

    \[\begin{tikzcd}[ampersand replacement=\&]
	{F^SS} \&\& {F^S} \\
	\\
	{Q F^T L S} \&\& {Q F^T L}
	\arrow["\varphi", from=1-3, to=3-3]
	\arrow["{Q \omega}"', from=3-1, to=3-3]
	\arrow["{\varphi_S}"', from=1-1, to=3-1]
	\arrow["{\epsilon^S_{F^S}}", from=1-1, to=1-3]
\end{tikzcd}\]
\end{lemma}
\begin{proof}
    \begin{align*}
        U^S Q \omega \circ U^S \varphi_S
        \\ = R U^T \epsilon^{F^T}_{F^T L} \circ RT \delta \circ \lambda_{L S} \circ S \eta^L_S
        \\ = R U^T \epsilon^{F^T}_{F^T L}  \circ \lambda_{T L} \circ S R \epsilon^L_{T L} \circ S \eta^L_{RTL} \circ S (\lambda_L \circ S \eta^L) && \text{(Naturality of the right $\lambda$ and $\eta$)} \\
        = R U^T \epsilon^{F^T}_{F^T L}  \circ \lambda_{T L} \circ S (\lambda_L \circ S \eta^L) && \text{(Triangle identity)}
        \\ = \lambda_L \circ S \eta^L \circ \mu^S && \text{(Composition identity and naturality of $\epsilon$)}
        \\ = U^S (\varphi \circ \epsilon^{S}_{F^S})
    \end{align*}
\end{proof}

\section{The adjoint lifting theorem}

We follow \cite[Chapter 4.5]{Borceux1994}. Consider the following situation, where $T, S$ are monads:

\[\begin{tikzcd}[ampersand replacement=\&]
	{C^T} \&\& {D^S} \\
	\\
	C \&\& D
	\arrow["Q", from=1-1, to=1-3]
	\arrow["R"', from=3-1, to=3-3]
	\arrow["{U^T}"', from=1-1, to=3-1]
	\arrow["{U^S}", from=1-3, to=3-3]
\end{tikzcd}\]

We shall prove that if $C^T$ has coequalizers, $R$ being a right adjoint implies $Q$ is a right adjoint. To start, suppose $R$ and $Q$ have right adjoints. Then without loss of generality we can suppose both adjoint squares commute:

\[\begin{tikzcd}[ampersand replacement=\&]
	{C^T} \&\& {D^S} \\
	\\
	C \&\& D
	\arrow["Q"', shift right, from=1-1, to=1-3]
	\arrow["R"', shift right, from=3-1, to=3-3]
	\arrow["{U^T}", shift left, from=1-1, to=3-1]
	\arrow["{U^S}", shift left, from=1-3, to=3-3]
	\arrow["{F^T}", shift left, from=3-1, to=1-1]
	\arrow["{F^S}", shift left, from=3-3, to=1-3]
	\arrow["P"', shift right, from=1-3, to=1-1]
	\arrow["L"', shift right, from=3-3, to=3-1]
\end{tikzcd}\]

We fix some notation:

\begin{itemize}
    \item $\eta^F, \epsilon^F$ are the unit/counit for $F \dashv G$

    \item $\eta^T, \mu^T$ are the unit and multiplication for the monad $T$
\end{itemize}

We'll always omit when there is no ambiguity.

Recall that we have the following coequalizer (generalizing a common algebraic construction):

\[\begin{tikzcd}[ampersand replacement=\&]
	{F^S S(D)} \&\& {F^S(D)} \&\& {(D, d)}
	\arrow["{\mu_{F^S(D)}}", shift left, from=1-1, to=1-3]
	\arrow["{F^S(d)}"', shift right, from=1-1, to=1-3]
	\arrow[from=1-3, to=1-5]
\end{tikzcd}\]

Then since $P$ preserves all colimits, passing the image we get an expression of $P(D, d)$ as a coequalizer. Now, since $P F_T = F^S L$, we start by retrieving $P \mu^S_{F^S}$. Thanks to Corollary \ref{thm:comorphism-monads-corollary} in the previous section, we already know hot to do it, by setting

$$\omega = \epsilon^{F^T}_{F^TL} \circ F^T \delta$$

We also set

$$\varphi = \pi_L \circ F^S \eta^L$$

Now, taking the coequalizer

\[\begin{tikzcd}[ampersand replacement=\&]
	{F^T L S (D)} \&\& {F^T L (D)} \&\& X
	\arrow["{\omega_D}", shift left, from=1-1, to=1-3]
	\arrow["{F^T L(d)}"', shift right, from=1-1, to=1-3]
	\arrow["x", from=1-3, to=1-5]
\end{tikzcd}\]

\noindent we have, as proved in the previous section,

\[\begin{tikzcd}[ampersand replacement=\&]
	{F^S S (D)} \&\& {F^S (D)} \&\& {(D, d) } \\
	\\
	{Q F^T L S(D)} \&\& {Q F^T L(D)} \&\& QX
	\arrow["{\epsilon^S_{F^S (D)}}", shift left, from=1-1, to=1-3]
	\arrow["{F^S (d)}"', shift right, from=1-1, to=1-3]
	\arrow["d", from=1-3, to=1-5]
	\arrow["{\varphi_{S(D)}}"', from=1-1, to=3-1]
	\arrow["{\varphi_D}"', from=1-3, to=3-3]
	\arrow["{Q \omega_D}", shift left, from=3-1, to=3-3]
	\arrow["{Q F^T L (d)}"', shift right, from=3-1, to=3-3]
	\arrow["Qx"', from=3-3, to=3-5]
	\arrow["\xi", dashed, from=1-5, to=3-5]
\end{tikzcd}\]

We prove $\xi$ does in fact satisfies the property of the unit arrow (recall in the previous section we had $\varphi = \eta^Q_{F^S}$).

\[\begin{tikzcd}[ampersand replacement=\&]
	{(D, d)} \&\& {Q(A, a) = (Ra, Ra \circ \lambda_A)}
	\arrow["y", from=1-1, to=1-3]
\end{tikzcd}\]

\noindent be given, so that

$$Ra \circ \lambda_A \circ Sy = y \circ d$$

For the sake of intuition, let us again suppose we have a suitable left adjoint $P$ for $Q$. Then one has

\begin{align*}
    \epsilon^{F^T} \circ F^T \epsilon^L_{U^T} \circ F^T L U^S (y)
    \\ = \epsilon^{Q} \circ P \epsilon^{F^S}_{Q} \circ F^T L U^S (y) 
    \\ = \epsilon^{Q} \circ P (y) \circ P \epsilon^S
    \\ = (y)^\flat \circ P(d)
\end{align*}

Thus we set

$$\chi = \epsilon^{F^T} \circ F^T \epsilon^L_{U^T} \circ F^T L U^S (y) = a \circ F^T \epsilon^L_{A} \circ F^T L U^S (y) $$

\noindent and verify it equalizes the arrows $\omega_D, F^T L (d)$:

\begin{align*}
    U^S \left( a \circ F^T \epsilon^L_{U^T} \circ F^T L U^S (y) \circ \epsilon^{F^T}_{F^T L (D)} \circ F^T \delta_{D} \right)
    \\ = U^S \left( a \circ \epsilon^{F^T}_{F^T U^T} \circ F^T \epsilon^L_{T U^T} \circ F^T L \lambda_{U^T} \circ F^T L S U^S (y) \right) && \text{(First naturality, then expand $\delta$ and triangle identity)}
    \\ = U^S (a) \circ \mu_A \circ T \epsilon^L_{TA} \circ T L \lambda_A \circ T L S U^S (y)
\end{align*}

On the other hand

\begin{align*}
    U^S \left( a \circ F^T \epsilon^L_{U^T} \circ F^T L U^S (y) \circ F^T L (d) \right)
    \\ = U^S \left( a \circ F^T \epsilon^L_{U^T} \circ F^T L R (a) \circ F^T L \lambda_{U^T} \circ F^T L S U^S (y) \right)
    \\ = U^S \left( a \circ \mu^T_A \circ F^T \epsilon^L_{TA} \circ F^T L \lambda_A \circ T L S U^S (y) \right)
\end{align*}
    
Hence we have a factorization

\[\begin{tikzcd}[ampersand replacement=\&]
	{F^T L S (D)} \&\& {F^T L (D)} \&\& X \\
	\\
	\&\&\&\& {(A, a)}
	\arrow["{\omega_D}", shift left, from=1-1, to=1-3]
	\arrow["{F^T L(d)}"', shift right, from=1-1, to=1-3]
	\arrow["x", from=1-3, to=1-5]
	\arrow["\chi"', from=1-3, to=3-5]
	\arrow["{y^\flat}", dashed, from=1-5, to=3-5]
\end{tikzcd}\]

We now verify $Q y^\beta \circ \xi \circ d = y \circ d$ (since $d$ is epi):

\begin{align*}
    U^S \left( y \circ d \right) = R U^T a \circ \lambda_A \circ S U^S (y)
\end{align*}

\begin{align*}
    U^S \left( Q y^\flat \circ \xi \circ d \right)
    \\ = U^S \left( Q \chi \circ \varphi_D \right)
    \\ = R U^T ( a \circ F^T \epsilon^L_{U^T} \circ F^T L U^S (y) ) \circ \lambda_{L(D)} \circ S \eta^L_{D}
    \\ = R U^T a \circ \lambda_A \circ S R \epsilon^T_A \circ S \eta^L_{A} \circ S U^S (y)
    \\ = R U^T a \circ \lambda_A \circ S U^S (y)
\end{align*}

Furthermore $Q f \circ \xi = Q g \circ \xi \iff Q (f x) \circ \varphi = Q (g x) \circ \varphi$

\noindent but upon expansion this becomes (in transpose notation)

$$U^S \flat^S \#^R \#^T (fx)$$

\noindent hence $fx = gx$ but $x$ is epi.

Packing together everything we've proved thus far, we have:

\begin{theorem}
    Let $\mathbb{D}$ be a $\mathcal{S}$-complete (resp. cocomplete) category. Then in the canonical $\mathrm{Cat}{(\mathcal{S})}$-indexing of $\mathbb{D}$, the transition functors have right (resp.left) adjoints. In particular, we have limit (colimit) functors $\mathrm{lim}_{\mathbb{C}}: \mathbb{D}^\mathbb{C} \to \mathbb{D}$
\end{theorem}

\section{Proving the theorem}

The indexed version of generator is pretty intuitive.
\begin{definition}
    An object $G \in C^I$ is called a generator if for $A, B \in C^J$, $f \neq g: A \to B$ we have a span

    \[\begin{tikzcd}
	K && J \\
	\\
	I
	\arrow["x"', from=1-1, to=3-1]
	\arrow["y", from=1-1, to=1-3]
    \end{tikzcd}\]
    
    \noindent and $h: x^\ast(G) \to y^\ast(A)$ s.t. $y^\ast(f) h \neq y^\ast(g) h$. We also have the dual notion called cogenerator.
    
\end{definition}

\begin{lemma}\label{thm:generator-epi-map}
    Suppose $\mathbb{C}$ is $\mathcal{S}$-cocomplete an locally small. Then $G \in \mathbb{C}^I$ is a separating family iff for every $A \in \mathbb{C}^J$ there's a span 

    
    \[\begin{tikzcd}
	K && J \\
	\\
	I
	\arrow["x"', from=1-1, to=3-1]
	\arrow["y", from=1-1, to=1-3]
    \end{tikzcd}\]

    \noindent and an epi $e: \Sigma_y x^\ast  (G) \twoheadrightarrow A$.
\end{lemma}
\begin{proof}
    One direction is obvious. Suppose $G$ is a separating family, and let $(v, w): K \to I \times J$ index morphisms $G \to A$ with universal arrow $h$, Now, if $f \neq g$, we have $h^\prime$ s.t. $y^\ast (f) h^\prime \neq y^\ast (g) h^\prime$, but $h^\prime = z^\ast (h)$, where $z$ is s.t. $(x,y) \circ z = (v, w)$, hence $w^\ast (f) h \neq w^\ast (g) h$. But since this holds for any $f, g$, clearly the transpose is epi.
\end{proof}


\begin{definition}[Indexed comma]
    Let $F: \mathbb{C} \to \mathbb{E}, G: \mathbb{D} \to \mathbb{E}$ be $\mathcal{S}$-indexed functors. Then $F \downarrow G$ is an $\mathcal{S}$-indexed category with

    \begin{enumerate}[i)]
        \item $\left( F \downarrow G \right)^I = F^I \downarrow G^I$

        \item For $h: F^IA \to G^IB$, $x: J \to I$,

        \[\begin{tikzcd}[ampersand replacement=\&]
	{F^IA} \&\& {G^IB} \&\& {F^IA} \&\& {F^IB} \\
	\& \downarrow \&\&\&\& \downarrow \\
	{x^\ast F^I A} \&\& {x^\ast G^I B} \&\& {F^J x^\ast A} \&\& {F^J x^\ast B} \\
	{F^J x^\ast A} \&\& {G^J x^\ast B} \\
	\&\&\&\& {x^\ast F^I A} \&\& {x^\ast F^I B}
	\arrow["h", from=1-1, to=1-3]
	\arrow["{x^\ast h}", from=3-1, to=3-3]
	\arrow["{\phi_G^x}", from=3-3, to=4-3]
	\arrow["{{\phi_F^x}^{-1}}", from=4-1, to=3-1]
	\arrow["{F^If}", from=1-5, to=1-7]
	\arrow["\cong"', from=3-5, to=5-5]
	\arrow["{F^J x^\ast f}", from=3-5, to=3-7]
	\arrow["{x^\ast F^If}", from=5-5, to=5-7]
	\arrow["\cong"', from=5-7, to=3-7]
\end{tikzcd}\]
    \end{enumerate}
\end{definition}

We assume $\mathcal{S}$ is finitely complete for the next results.

The following is tedious but straightforward from the definitions:

\begin{lemma}
    If $F: \mathbb{C} \to \mathbb{D}$ is $\mathcal{S}$-continuous, the indexed comma category $B \downarrow F$ (where $B \in \mathbb{C}^1$) is $\mathcal{S}$-complete.
\end{lemma}


\begin{lemma}
    If $\mathbb{C}$ is locally small, $B \downarrow F$ is also locally small.
\end{lemma}
\begin{proof}
    The prime candidate for a generic morphism between the objects comprising the span

    \[\begin{tikzcd}[ampersand replacement=\&]
	{I^\ast B} \& {F^I Y} \\
	{F^I X}
	\arrow["f"', from=1-1, to=2-1]
	\arrow["g", from=1-1, to=1-2]
\end{tikzcd}\]

    \noindent would be $F^I h$, where $h$ is generic for morphisms $X \to Y$ and we have $d: J \to I \times I$. The problem is that the corresponding triangle needn't commute. Therefore we use \ref{thm:locally-small-def-eq-inv} i), since we are assuming $\mathcal{S}$ is cartesian, we take $z: K \rightarrowtail J$ generic w.rt. making the triangle commute. Then $d \circ z$ is the desired indexing morphism.
\end{proof}

\begin{lemma}
    If $\mathbb{C}$ is well-powered and $F$ is continuous, $B \downarrow F$ is also well-powered.
\end{lemma}
\begin{proof}
    We want a morphism generic for subobjects of $X$ for which $f: I^\ast B \to F^I X$ factors through $F^I A \rightarrowtail F^I X$. Let $m: A \to m^\ast X$ be the generic suboject with $x: J \to I$ indexing. We take the pullback

    \[\begin{tikzcd}[ampersand replacement=\&]
	D \&\& {F^I A} \\
	\\
	{J^\ast B} \&\& {F^J x^\ast X}
	\arrow[from=3-1, to=3-3]
	\arrow[tail, from=1-3, to=3-3]
	\arrow["h"', from=1-1, to=3-1]
	\arrow[from=1-1, to=1-3]
	\arrow["\lrcorner"{anchor=center, pos=0.125}, draw=none, from=1-1, to=3-3]
\end{tikzcd}\]
    
    \noindent and then use the fact invertibility is definable in $\mathbb{D}$ bt \ref{thm:locally-small-def-eq-inv} ii), to get a morphism $z: K \rightarrowtail J$ universal for morphisms which take $h$ to an iso. Then  $x \circ z:K \to I$ is the desired indexing map.
\end{proof}


We also have:

\begin{comment}
\begin{lemma}
    Suppose $\mathbb{C}$ is $\mathcal{S}$-cocomplete and locally small. Then $G \in \mathcal{C}^I$ is a generator
\end{lemma}
\end{comment}

\begin{theorem}
    If $\mathbb{C}$ admits a cogenerator object, so does $F \downarrow B$.
\end{theorem}
\begin{proof}
    The proof is quite analogous to the non-indexed case. Let $G$ be the generator and let $k: J^\ast B \to x^\ast F^I G$ be a generic arrow for morphisms $B \to F^I G$. We wish to prove $J^\ast B \to x^\ast F^I X \cong F^J x^\ast X$ is the desired generator object. Let us have two different arrows

    \[\begin{tikzcd}[ampersand replacement=\&]
	\& {F^K X} \\
	{K^\ast B} \\
	\& {F^K Y}
	\arrow["a", from=2-1, to=1-2]
	\arrow["b"', from=2-1, to=3-2]
	\arrow["{F^K f}"', shift right=2, from=1-2, to=3-2]
	\arrow["{F^K g}", shift left=2, from=1-2, to=3-2]
\end{tikzcd}\]

Then we have

\[\begin{tikzcd}[ampersand replacement=\&]
	L \&\& K \\
	\\
	J
	\arrow["x"', from=1-1, to=3-1]
	\arrow["y", from=1-1, to=1-3]
\end{tikzcd}\]

and $h: y^\ast Y \to x^\ast G$ s.t. $h y^\ast (f) \neq h y^\ast (g)$. Thus we have a morphism

\[\begin{tikzcd}[ampersand replacement=\&]
	\&\& {F^K y^\ast Y} \\
	{L^\ast B} \\
	\&\& {F^K x^\ast G}
	\arrow["{y^\ast b}", from=2-1, to=1-3]
	\arrow["{F^K h}", from=1-3, to=3-3]
	\arrow["{u^\ast k}"', from=2-1, to=3-3]
\end{tikzcd}\]

\noindent (modulo coherence isos).

\end{proof}

Before moving on to the first big proof, we need some lemmas:

\begin{lemma}
    Let $\mathbb{D}$ be an internal category with a weakly terminal object (non-uniqueness). Then $\mathbb{D}$ is connected, that is,

    \[\begin{tikzcd}[ampersand replacement=\&]
	{D_1} \&\& {D_0} \&\& {\pi_0(\mathbb{D}) = 1}
	\arrow["{d_0}", shift left, from=1-1, to=1-3]
	\arrow["{d_1}"', shift right, from=1-1, to=1-3]
	\arrow[two heads, from=1-3, to=1-5]
\end{tikzcd}\]
\end{lemma}
\begin{proof}
    Having a weakly terminal object means

    \[\begin{tikzcd}[ampersand replacement=\&]
	{C_1} \&\& {C_0 \times C_0} \\
	\\
	\&\& X
	\arrow["{(x, \top)}", from=3-3, to=1-3]
	\arrow["\exists", dashed, from=3-3, to=1-1]
	\arrow[from=1-1, to=1-3]
\end{tikzcd}\]

 But then the coequalizer arrow equalizes $x$ and $\top$, meaning this is the unique such arrow, which always exists, so $\pi_0(\mathbb{D})$ is terminal.
\end{proof}

\begin{theorem}
    Let $\mathbb{D}$ be a connected internal category in $\mathcal{S}$ which admits an arrow $1 \to D_0$, and $\mathbb{C}$ an $\mathcal{S}$-indexed locally small category. Then the constant diagram functor $\mathbb{D}^\ast: \mathbb{C} \to \mathbb{C}^\mathbb{D}$ is fully faithful.
\end{theorem}
\begin{proof}
    Faithfulness is straightforward.

    Let $f: D_0^\ast A \to D_0^\ast B$. Let $g: J^\ast A \to J^\ast B$ classify morphisms $A \to B$. Then we have $f \cong u^\ast (g)$, but since $d_0^\ast(f) = d_1^\ast(f)$ (as the constant diagrams have identities for maps), we have that $u d_0 = u d_1$, inducing, by connectedness, a factorization $u = x \circ \top_{D_0}$, but then we just proved $f = Q_0^\ast (x^\ast (g) )$. We leave it to the reader to parse this statement in light of the existence of coherence isos, but we promise everything works out.
\end{proof}

\begin{theorem}[The special indexed initial object theorem]
    Let $\mathcal{S}$ be cartesian. Let $\mathbb{C}$ be complete, locally small, well-powered and with a cogenerator. Then $\mathbb{C}$ has an indexed initial object.
\end{theorem}
\begin{proof}
    Let us first recall the non-indexed version: the desired initial object is the intersection of all the subobjects of the product of such cogenerating family.
    
    It suffices to prove it for $\mathcal{C}^1$. Let $m: Y \to U^\ast(X)$ be the generic subobject of $X = \Pi_I (G)$. Now, we take the intersection
    
    \[\begin{tikzcd}[ampersand replacement=\&]
	{\pi_1^\ast (Y) \cap \pi_2^\ast (Y)} \&\& {\pi_1^\ast (Y)} \\
	\\
	{\pi_2^\ast (Y)} \&\& {{U \times U}^\ast (X)}
	\arrow["{\pi_2^\ast (m) = b}"', from=3-1, to=3-3]
	\arrow["{\pi_1^\ast (m) = a}", from=1-3, to=3-3]
	\arrow[from=1-1, to=3-1]
	\arrow[from=1-1, to=1-3]
	\arrow["{a \cap b = z^\ast(m)}", from=1-1, to=3-3]
    \end{tikzcd}\]

    \noindent which is again classified by some map $z: U \times U \to U$. Then we take the equalizer

    \[\begin{tikzcd}[ampersand replacement=\&]
	{U_1} \&\& {U \times U} \&\& U
	\arrow["z", shift left, from=1-3, to=1-5]
	\arrow["{\pi_1}"', shift right, from=1-3, to=1-5]
	\arrow["h", tail, from=1-1, to=1-3]
    \end{tikzcd}\]

    If the reader is looking for intuition, consider again the naive $\mathrm{Set}$-indexing: $z$ is the intersection map in $U = \mathrm{Sub}(A)$, and $U_1$ is precisely the subset representing the relation $\leq$. We thus have a category $\mathbb{U}$ with $d_0 = \pi_1 h, d_1 = p_2 h$. In fact, it is an internal poset since $h$ is mono.

    Now, consider the following diagram structure $(Y, \Phi)$ on $\mathbb{C}^\mathbb{U}$: $\Phi = (h \pi_1)^\ast (Y) \to (h \pi^2)^\ast (Y)$ is precisely the image under $h^\ast$ of the left leg on the intersection pullback, which $h$ collapses to a triangle.
    Furthermore, consider the constant diagram on $X$,  $\mathbb{U}^\ast X$. Then the universal mono $m: Y \to U^\ast (X)$ induces a morphism of diagrams $(Y, \Phi) \rightarrowtail \mathbb{U}^\ast X$, which is again mono because the forgetful functor on diagrams is faithful. Then we pass the limit functor $\mathrm{lim}_\mathbb{U}$ on this arrow, which is a right adjoint and gives us a mono arrow $B \rightarrowtail A$.

    Now, by the results above, we have $A \cong X$, so we actually have an arrow $k: B \rightarrowtail X$. Furthermore, consider the unit $\eta: 1 \implies \Pi_{U \to \mathbb{U}} (U \to \mathbb{U})^\ast$. It is clearly mono since $(U \to \mathbb{U})^\ast$ is clearly faithful, which means we obtain a a mono arrow $h = \mathrm{lim}_{\mathbb{U}} \eta_{(Y, \Phi)}: B = \mathrm{lim}_{\mathbb{U}}(Y, \Phi) \to \Pi_U (Y)$ (intuitively this expresses the limit as a subobject of the product).

    Now, if we take the transpose $h^\prime: U^\ast(B) \to Y$, we have the following commutative diagram

    \[\begin{tikzcd}[ampersand replacement=\&]
	{U^\ast(B)} \&\& Y \\
	\\
	\&\& {U^\ast(X)}
	\arrow["{U^\ast (k)}"', from=1-1, to=3-3]
	\arrow["{h^\prime}", from=1-1, to=1-3]
	\arrow["m", from=1-3, to=3-3]
    \end{tikzcd}\]
    
    \noindent since (we omit coherence isos and write $[m]$ for the induced arrow in $\mathbb{C}^\mathbb{U}$, $I$ for the functor $U_0 \to \mathbb{U}$):

    $$k = {\eta^{\mathbb{U}}}^{-1}_{X} \circ \mathrm{lim}_{\mathbb{U}}{([m])}$$

    \begin{align*}
        m \circ h^\prime
        \\ = I^\ast ([m]) \circ \epsilon^U \circ U^\ast \mathrm{lim}_{\mathbb{U}} \eta^{I}_Y
        \\ = \epsilon^U_{U^\ast} \circ U^\ast \mathrm{lim}_{\mathbb{U}} \Pi_I I^\ast ([m]) \circ U^\ast \mathrm{lim}_{\mathbb{U}} \eta^{I}_Y
        \\ = \epsilon^U_{U^\ast} \circ  U^\ast \mathrm{lim}_{\mathbb{U}} \eta^{I}_{\mathbb{U}^\ast X} \circ {\eta^{\mathbb{U}}}_{X} \circ {\eta^{\mathbb{U}}}^{-1}_{X} \circ \mathrm{lim}_{\mathbb{U}}{([m])}
        \\ = {\eta^{\mathbb{U}}}^{-1}_{X} \circ \mathrm{lim}_{\mathbb{U}}{([m])} = k
    \end{align*}

    Therefore we conclude $k: B \rightarrowtail X$ is in fact the minimal subobject.  We now prove that $B$ is indeed the desired initial object of $\mathbb{C}^1$. Let $C \in \mathbb{C}^1$. Then by the dual of Lemma \ref{thm:generator-epi-map}, we have $x: J \to I$ and a mono $C \rightarrowtail \Pi_J x^\ast (G)$. Since, $\Pi_J \cong \Pi_I \Pi_x$, we form the pullback

    \[\begin{tikzcd}[ampersand replacement=\&]
	B \\
	U \&\& {\Pi_I(G) = X} \\
	\\
	C \&\& {\Pi_J x^\ast(G)}
	\arrow["{\Pi_I \eta^x_G}", from=2-3, to=4-3]
	\arrow[tail, from=4-1, to=4-3]
	\arrow[from=2-1, to=4-1]
	\arrow[tail, from=2-1, to=2-3]
	\arrow[tail, from=1-1, to=2-1]
	\arrow[from=1-1, to=2-3]
	\arrow["\lrcorner"{anchor=center, pos=0.125}, draw=none, from=2-1, to=4-3]
\end{tikzcd}\]

    Furthermore, clearly any coequalizer from $B$ must be an isomorphism by minimality, hence $B$ is in fact an initial object.
\end{proof}

Finally we can prove the desired theorem, but first we need another useful lemma.

\begin{lemma}
    Let $\mathcal{S}$ have finite products. Then the change of base functor $\Sigma_I^\ast: \mathrm{Cat}_\mathcal{S} \to \mathrm{Cat}_\mathcal{S /I}$ preserves the property of having a (co) generator.
\end{lemma}
\begin{proof}
    Let $G \in \mathbb{C}^J$. We prove that $\pi_J^\ast (G)$ is the desired generator in the new category. Let $f \neq g$ in $\mathbb{C}^U, u: U \to I$. Then we have

    \[\begin{tikzcd}[ampersand replacement=\&]
	K \&\& U \&\& K \&\& {J \times I} \\
	\&\&\& \rightarrow \\
	J \&\&\&\& U \&\& I
	\arrow["{\pi_I}", from=1-7, to=3-7]
	\arrow["u"', from=3-5, to=3-7]
	\arrow["x"', from=1-1, to=3-1]
	\arrow["y", from=1-1, to=1-3]
	\arrow["{(x, uy)}", from=1-5, to=1-7]
	\arrow[from=1-5, to=3-5]
	\arrow["uy"', from=1-5, to=3-7]
\end{tikzcd}\]
\end{proof}

Furthermore, such operation clearly preserves the other properties we need.

\begin{theorem}[The special indexed adjoint functor theorem]
    Let $\mathbb{S}$ be cartesian, and $\mathbb{C}, \mathbb{D}$ $\mathcal{S}$-indexed categories which are locally small and complete, and $\mathbb{C}$ $\mathcal{S}$-complete. Then $F: \mathbb{C} \to \mathbb{D}$ has a left adjoint iff it is continuous.
\end{theorem}
\begin{proof}
    We apply the previous theorem on each instance $B \downarrow F$, so that we may construct $H^1 \dashv F^1$. Furthermore, we can apply chance-of-base on the previous lemma to consider categories $B \downarrow F$ where $B \in \mathbb{D}^I$ originally. Now, we have $H^I \dashv F^I$ componentwise, but we need an \textbf{indexed} adjunction. To this end, we invoke the observation on section 5 that it suffices for the mate of $F^x$ to be invertible, but this follows easily from $\mathcal{S}$-continuity if we take all left adjoints of the mate (see Definition \ref{def:indexed-continuous-functor}).
\end{proof}

\nocite{*}
\printbibliography

\end{document}
