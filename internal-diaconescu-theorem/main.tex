\documentclass[10pt, oneside]{article} 
\usepackage{amsmath, amsthm, amssymb, calrsfs, wasysym, verbatim, bbm, color, graphics, geometry}

\usepackage{eucal}[mathcal]

\usepackage[backend=biber]{biblatex}
\addbibresource{references.bib}
\usepackage{quiver}

\usepackage{abstract}
\renewcommand{\abstractname}{}    % clear the title
\renewcommand{\absnamepos}{empty}

\geometry{tmargin=.75in, bmargin=.75in, lmargin=.75in, rmargin = .75in}  

\newcommand{\R}{\mathbb{R}}
\newcommand{\C}{\mathbb{C}}
\newcommand{\Z}{\mathbb{Z}}
\newcommand{\N}{\mathbb{N}}
\newcommand{\Q}{\mathbb{Q}}
\newcommand{\Cdot}{\boldsymbol{\cdot}}

\newtheorem{theorem}{Theorem}[section]
\newtheorem{definition}[theorem]{Definition}
\newtheorem{remark}[theorem]{Remark}
\newtheorem{lemma}[theorem]{Lemma}
\newtheorem{corollary}[theorem]{Corollary}
\newtheorem{example}[theorem]{Example}

\title{Diaconescu's theorem for internal sheaves}
\author{Gabriel C. Barbosa}
\date{2023}

\begin{document}

\maketitle

\tableofcontents

\vspace{.25in}

\section{Background}

We strongly recommend the reading of the first notes by the author titled \textit{The indexed adjoint functor theorem}.

\section{Three ways of seeing universe functors}

We fix the following notation for internal categories

\begin{itemize}
    \item $c$ is composition
    \item $d_0, d_1$ are source and target
    \item $e$ is the identity map
    \item $p_0, p_1$ are the projections from $C_2 = C_1 \times_{C_0} C_1$
\end{itemize}

Following \cite{MacLane1994}, one can define a functor in universe using the notion of left $\mathbb{C}$-object.

\begin{definition}
    Let $\mathcal{S}$ be a category with pullbacks and $\mathbb{C}$ a category internal to $\mathcal{S}$. A left $\mathbb{C}$-object is a map $\pi: F \to C_0$ equipped with an action $\mu: F \times_{C_0} C_1 \to F$ (where the pullback is over $d_0$) s.t. the following commutes

    \[\begin{tikzcd}[ampersand replacement=\&]
	{F \times_{C_0} C_1} \&\& {C_1} \&\& F \&\& {F \times_{C_0} C_1} \\
	\\
	F \&\& {C_0} \&\&\&\& F \\
	{F \times_{C_0} C_1 \times_{C_0} C_1} \&\& {F \times_{C_0} C_1} \\
	\\
	{F \times_{C_0} C_1} \&\& F
	\arrow["{p_1}", from=1-1, to=1-3]
	\arrow["\mu"', from=1-1, to=3-1]
	\arrow["\pi"', from=3-1, to=3-3]
	\arrow["{d_1}", from=1-3, to=3-3]
	\arrow["{1 \times e}", from=1-5, to=1-7]
	\arrow["{1_F}"', no head, from=1-5, to=3-7]
	\arrow["\mu", from=1-7, to=3-7]
	\arrow["{1 \times c}", from=4-1, to=4-3]
	\arrow["{\mu \times C_1}"', from=4-1, to=6-1]
	\arrow["\mu", from=4-3, to=6-3]
	\arrow["\mu"', from=6-1, to=6-3]
\end{tikzcd}\]

Morphisms between actions are defined in the obvious way.
\end{definition}

We will refer to $\mathbb{C}$-actions from now on as $\mathbb{C}$-copresheaves. It is a standard result from \cite{MacLane1994} that

\begin{theorem}
    The category $\mathrm{Copresh}{(\mathbb{C})}$ is a topos and we have an adjunction between $\mathcal{S}/C_0$ and $\mathrm{Copresh}{(\mathbb{C})}$ given by the right adjoint forgetful functor and right adjoint the free action functor.
\end{theorem}

But first,

\begin{definition}
    We have an adjunction between $\mathcal{D}^{C_0}$ and $\mathcal{D}^\mathbb{C}$ (where the notation refers to non-indexed structures).
\end{definition}

\begin{theorem}
    Let $\mathbb{D}$ have $\mathcal{S}$-indexed products (coproducts), then $\mathcal{D}^\mathbb{C}$ is comonadic (respectively monadic). Moreover, $\mathbb{D}^\mathbb{C}$ has $\mathcal{S}$-indexed products (coproducts respectively).
\end{theorem}
\begin{proof}
    Consider the functor
    
\end{proof}

We can actually consider such actions as functors. Not only any functor, but a discrete opfibration, as we'd assume from the analogy with the category of elements fibration from which the definition above draws upon.

\begin{definition}
    A discrete opfibration is an internal functor $F: \mathbb{F} \to \mathbb{C}$ such that

    \[\begin{tikzcd}[ampersand replacement=\&]
	{F_1} \&\& {C_1} \\
	\\
	{F_0} \&\& {C_0}
	\arrow["{d_1}", from=1-1, to=3-1]
	\arrow["{F_0}", from=3-1, to=3-3]
	\arrow["{F_1}"', from=1-1, to=1-3]
	\arrow["{d_1}"', from=1-3, to=3-3]
\end{tikzcd}\]

    \noindent is a pullback square
\end{definition}

To go from left $\mathbb{C}$-actions to discrete opfibrations, we consider $F_0 = F, F_1 = F \times_{C_0} C_1$, and  $d_0 = p_0, d_1 = \mu, c = 1 \times c$ (slight abuse of notation here), then set $F_0 = \pi, F_1 = p_1$. Converse is straightforward by definition.

Thirdly, we can consider the notion of internal diagrams.

\begin{definition}
    Let $\mathbb{E}$ be a $\mathcal{S}$-indexed category. A $\mathbb{C}$-shaped diagram in $\mathbb{E}$ consists of the following data:

    \begin{itemize}
        \item An object $P \in \mathbb{E}^{C_0}$
        \item An arrow $\Phi: d_0^\ast P \to d_1^\ast P$
    \end{itemize}

    such that $e^\ast \Phi = 1_P$ and $c^\ast \Phi = p_0^\ast \Phi \circ p_1^\ast \Phi$ modulo the coherence isos. Morphisms between such diagrams are defined in the obvious way.

    We write $\mathbb{E}^\mathbb{C}$ for the corresponding category.
\end{definition}

It is a long but straightforward task to show that $\mathbb{C}$-diagrams in $\mathbb{S}$ ($\mathcal{E}$ indexed over itself) correspond to $\mathbb{C}$-copresheaves, but also:

\begin{lemma}
    Let $\mathbb{C}$ be internal to $\mathbb{S}$. Then $\mathbb{S}^\mathbb{C} \cong [\underline{\mathbb{C}}, \mathbb{S}]$. Given any indexed functor $F$, we define a diagram with $P = F^{C_0}(1_{C_0}), \Phi = F^{C_1}(1_{C_1})$. Conversely, any diagram $(P, \Phi)$ induces $F$ with $F(x: I \to C_0) = x^\ast P, F(y: J \to C_1) = y^\ast \Phi$.
\end{lemma}

Thus we have 3 different ways of looking at the notion of copresheaf and hence of presheaf internally.

\begin{lemma}
    Let $p: \mathcal{S} \to \mathbb{E}$ be a a pullback-preserving functor, $\mathbb{C}$ a category internal to $\mathcal{S}$ and $\mathbb{D}$ a $\mathcal{E}$-indexed category. Then $[p(\mathbb{C}), \mathbb{D}] \simeq [\mathbb{C}, p^\ast(\mathbb{D})]$.
\end{lemma}

We can consider a $\mathcal{S}$-indexed category of diagrams, by setting $(\mathbb{D}^\mathbb{C})^I = (\mathbb{D})^{\mathbb{C} \times I} \cong (\mathbb{D}^I)^\mathbb{C}$.

\begin{remark}
    Consider the functor $P$ taking $I \in \mathcal{S}$ to the discrete path-action
    
    $$(p_1: I \times C_0 \to C_0, 1 \times d_1: I \times C_1 \to I \times C_0) \; .$$
    
    It is easy to see that $\mathrm{CoPresh}{(\mathbb{C})}/P(I) \cong \mathrm{CoPresh}{(\mathbb{C} \times I)}$. 
\end{remark}

It is pretty easy to see that the free action functor yields a functor $\underline{\mathbb{C}} \to \mathrm{Presh}{(\mathbb{C})}$. However, the problem is that we wish for a cartesian functor, or equivalently a $\mathcal{S}$-indexed functor. Thus we replace $\mathrm{Copresh}{(\mathbb{C})}$ by its indexed version $\mathbb{S}^\mathbb{C}$ that we just defined. However, to make it work, we need actually consider our action $I \times_{C_0} C_1 \to C_0$ as being over $\mathbb{C} \times I$ instead, in the trivial way, that is, we pass from $\pi: I \times_{C_0} C_1 \to C_0$ to $1 \times \pi: I \times_{C_0} C_1 \to I \times C_0$. And thus we have our Yoneda functor for indexed categories.

\begin{remark}
    Using the previous remark we can easily prove that $[\mathbb{C}, \mathcal{S}]^\mathbb{D} \cong [\mathbb{C} \times \mathbb{D}, \mathcal{S}]$.
\end{remark}

With the above remark we can more simply define the Yoneda functor as the profunctor $(d_0, d_1): C_1 \to C_0 \times C_0$ with action by $\mathbb{C}^{op} \times \mathbb{C}$.

\section{Internal free cocompletion}

Following the spirit of analogy with $\mathrm{Set}$, let us consider the colimit functor as taking the connected components of its diagram.

\begin{definition}
    Let $\mathbb{C}$ be an internal category. We define the connected components $\pi_0 \mathbb{C}$ as being the coequalizer of $d_0, d_1: C_1 \rightrightarrows C_0$.
\end{definition}

It is easy to see this functor is a left adjoint to the discrete category functor.

\begin{lemma}
    Let $\Delta: \mathbb{S} \to [\mathbb{C}, \mathbb{S}]$ be the constant diagram functor. Then it has a left adjoint which acts on discrete opfibrations $\mathbb{F} \to \mathbb{C}$ by taking them to $\pi_0 \mathbb{F}$.
\end{lemma}
\begin{proof}
    The functor $\Delta$ is the composition of the discrete functor and the pullback. Thus its left adjoint is $\pi_0 \circ U$.
\end{proof}

More generally, for any pullback-preserving functor $p: \mathcal{S} \to \mathcal{E}$, we have the analogous construction since $\mathbb{E}^\mathbb{C} \cong  \mathbb{E}^{p{(\mathbb{C})}}$.


\begin{lemma}
    Taking colimits in diagrams amounts to the following operation  in copresheaves, up to the aforementioned equivalences:

    \[\begin{tikzcd}[ampersand replacement=\&]
	{F \times_{C_0} C_1} \& F \& {F^\prime} \&\& {(F \times_{C_0} C_1) \times_{D_0} D_1} \& {F \times_{D_0} D_1} \& {F^\prime \times_{D_0} D_1} \\
	\&\& {D_0} \&\&\&\& {F^\prime}
	\arrow["{\mu_C}", shift left=1, from=1-1, to=1-2]
	\arrow["{p_F}"', shift right=1, from=1-1, to=1-2]
	\arrow["f", from=1-2, to=1-3]
	\arrow["{\pi_D}"', from=1-2, to=2-3]
	\arrow["{\pi^\prime}", dashed, from=1-3, to=2-3]
	\arrow["{\mu_C \times 1}", shift left=1, from=1-5, to=1-6]
	\arrow["{f\mu_F}"', from=1-6, to=2-7]
	\arrow["{p_F \times 1}"', shift right=1, from=1-5, to=1-6]
	\arrow[from=1-6, to=1-7]
	\arrow["{\mu^\prime}", dashed, from=1-7, to=2-7]
\end{tikzcd}\]
    
    \noindent recalling that in topoi pullbacks preserve coequalizers.
\end{lemma}

\begin{theorem}
    Let $\mathbb{C}$ be internal in $\mathcal{S}$, and $F$ a $\mathbb{C}$-shaped diagram in $\mathbb{S}$ corresponding to a discrete opfibration $\gamma: \mathbb{F} \to \mathbb{C}$. There exists a diagram $D$ of shape $\mathbb{F}^\mathrm{op}$ in $[\mathbb{C}, \mathcal{S}]$ whose colimit is precisely $F$.
\end{theorem}
\begin{proof}
    We diverge from the Elephant's prove using the above lemma. Keeping the concrete case in mind for intuition, the desired presheaf is given by 

    \[\begin{tikzcd}[ampersand replacement=\&]
	{F \times_{C_0}C_1} \&\& {(F \times_{C_0} C_1) \times_{C_0} C_1 \times_{C_0} C_1} \&\& {(F \times_{C_0} C_1) \times C_1} \\
	\\
	{F \times C_0} \&\& {F \times_{C_0}C_1} \&\& {F \times C_0}
	\arrow["{\pi = 1 \times d_1}", from=1-1, to=3-1]
	\arrow["{\mu_F \times d_0}", from=1-5, to=3-5]
	\arrow["{1 \times d_1}"', from=3-3, to=3-5]
	\arrow["\lrcorner"{anchor=center, pos=0.125}, draw=none, from=1-3, to=3-5]
	\arrow["{\mu \times p_1}"', from=1-3, to=3-3]
	\arrow["{1 \times p_0}", from=1-3, to=1-5]
\end{tikzcd}\]

\[\begin{tikzcd}[ampersand replacement=\&]
	{F \times_{C_0} C_1 \times_{C_0} C_1 \times_{C_0} C_1} \&\& {F \times_{C_0}C_1}
	\arrow["{\mu = 1 \times c_3}", from=1-1, to=1-3]
\end{tikzcd}\]

\noindent where $c_3$ is the double composition $C_3 \to C_1$.

We then have

\[\begin{tikzcd}[ampersand replacement=\&]
	{F \times_{C_0} C_2} \&\& {F \times_{C_0} C_1} \&\& F \\
	\&\&\&\& D
	\arrow["{1 \times c}"', shift right=1, from=1-1, to=1-3]
	\arrow["{(p_0, p_1)}", shift left=1, from=1-1, to=1-3]
	\arrow["{p_F}", from=1-3, to=1-5]
	\arrow["a"', from=1-3, to=2-5]
	\arrow["{a (1, e \pi_F)}", dashed, from=1-5, to=2-5]
\end{tikzcd}\]

\noindent so $\pi^\prime = \pi_F$ and likewise for the pulled back diagram. A painful calculation involving the right adjoint of pullback  to show $\mu^\prime = \mu_F$ can be avoided by resorting to uniqueness.
\end{proof}

We are halfway towards the definition of $\_ \otimes F$, but first we need to consider colimits in general, and not just the particular case of $\mathcal{S}$-valued diagrams.

Recall that a cocomplete fibration is equivalently a fibration $p: X \to P$ s.t. $X$ is cocomplete and $p$ cocontinuous. Equivalenly, the corresponding indexed category has all categories cocomplete and functors cocontinuous.

Given an object $X \in \mathbb{E}^I$, we can define a diagram $\Delta X$ with $P = (I \times C_0 \to I)^\ast X, \Phi = 1_{(I \times C_1)^\ast X}$. Equivalently, this is an indexed functor $\underline{\mathbb{C} \times I} \to \mathbb{E}$ taking (up to coherent isomorphisms) $(a,  b): J \to C_0 \times I$ to $b^\ast X$ and $(x, y): K \to C_1 \times I$ to $1_{y^\ast X}$. Regarding it as an indexed functor $\underline{\mathbb{C}} \to \mathbb{E}^I$ instead, we have $F(x: J \to C_0) = (J \times I)^\ast X$ up to iso. Given any functor in $\underline{\mathbb{C}} \to \mathbb{E}^I$, take its colimit on the component $1$. This operation is clearly functorial as well in $[\mathbb{C}, \mathbb{E}]$.

Let $F$ be one such indexed functor, and $f: \operatorname{colim}(F) \to X \in \mathbb{E}^I$ be given. Then this clearly induces a cocone for $X$, which in turn corresponds to the indexed transformation componentwise given by $\mu_X: F(x: 1 \to C_0) \to X$ where $\mu_X$ is the cocone leg up to isomorphism. The other components are obtained by simply taking the pseudofunctorial action.

\begin{theorem}
    If $\mathbb{E}$ is a cocomplete $\mathcal{S}$-indexed category and $\mathbb{C}$ is internal to $\mathcal{S}$, the functor $\Delta: \mathbb{E} \to [\underline{\mathbb{C}}, \mathbb{E}]$ has a left adjoint $\operatorname{colim}$ obtained by taking the colimit at $\mathbb{E}^I$.
\end{theorem}

Uniqueness of adjoints tells us this definitions agrees with the one on $[\mathbb{C}, \mathcal{S}]$.

Now we are ready to define the tensor functor and state the final result of this section:

\begin{theorem}
    Let $\mathbb{D}$ be an indexed cocomplete category. Then we have an equivalence of categories between $[\underline{\mathbb{C}}, \mathbb{D}]$ and the category of indexed cocontinuous functors $[\mathbb{C}^{op}, \mathcal{S}] \to \mathbb{D}$.
\end{theorem}
\begin{proof}
    One side of the equivalence is given by precomposition with the previously defined Yoneda functor $Y: \mathbb{C} \to [\mathbb{C}^{op}, \mathcal{S}]$.

    For the other side, let $F: \underline{\mathbb{C}} \to \mathbb{D}$ be given. Note that we need only define a cocontinuous functor agreeing with $F$ on the representables, just as in the concrete case, since we already proved every . Let us look at the preshaves as discrete fibrations $p: \mathbb{F} \to \mathbb{C}$. We then define $p \otimes F$ as $\operatorname{colim}(F \circ p: \underline{\mathbb{F}} \to \underline{\mathbb{C}} \to \mathbb{D})$. We can easily extend this definition to an indexed version.

    Now given a representable $Y(x: I \to C_0)$, note that the externalization of its fiber is actually $\underline{\mathbb{C}}/x$, and so, imitating the same argument we'd do in the concrete case, we conclude the colimit is $F(x)$. Cocontinuity is given by commutativity of colimits, and we conclude the proof.
\end{proof}

\section{Toposes over a base}

\section{The theorem on presheaves}

Following the concrete analogues, we now aim to prove that presheaves with filtered fiber domain yield under tensoring exact functors. We need thus to develop a suitable internal notion. As usual, we follow the Elephant.



\section{Internal sites}

\section{Diaconescu's Theorem}

\nocite{*}
\printbibliography

\end{document}
